\usepackage[utf8]{inputenc}

\usepackage[linesnumbered,ruled,vlined]{algorithm2e}

\usepackage{amsbsy}

\usepackage{amscd}

\usepackage{amsfonts}

\usepackage{amsmath}

\usepackage{amssymb}

\usepackage{amstext}

\usepackage{amsthm}

% Manages culturally-determined typographical rules for a
% wide range of languages.
\usepackage[spanish, english]{babel}
% https://tex.stackexchange.com/questions/82009/babel-and-the-decimal-separator
\decimalpoint

% Defines commands to access bold math symbols.
% \usepackage{bm}

% Offers customization of captions in floating environments.
% \usepackage{caption}

% Provides both foreground and background color management.
\usepackage{color}

% Provides advanced facilities for in-line and display quotations.
\usepackage{csquotes}

% Deprecated. Use graphicx package instead.
% \usepackage{epsfig}

\usepackage{euscript}

% Provides extensive facilities, both for constructing headers and footers,
% and for controlling their use.
\usepackage{fancyhdr}

% Defines floating objects such as figures and tables.
\usepackage{float}

% Allows the user to select font encodings.
\usepackage[T1]{fontenc}

% Creates regions that can break across pages.
\usepackage{framed}

% Reduces horizontal margin.
% \usepackage{fullpage}

% Provides an easy and flexible user interface to customize page layout,
% implementing auto-centering and auto-balancing mechanisms.
\usepackage[lmargin=2.5cm, rmargin=2.5cm, tmargin=3.0cm, bmargin=3.0cm]{geometry}

% Enhances graphics package.
\usepackage{graphicx}

\usepackage{hyperref}

% Produces multiple indexes.
\usepackage{imakeidx}

% Translates various standard and other input encodings.
\usepackage[utf8]{inputenc}

% Enhances LaTeX cross-referencing.
\usepackage[nameinlink, spanish]{cleveref}

\usepackage{latexsym}

\usepackage{mathtools}

% Simplifies the inclusion of external multi-page PDF documents in LaTeX documents.
\usepackage{pdfpages}

% Replaces strings in encapsulated PostScript figures.
\usepackage{psfrag}

% Allows figures or tables to have text wrapped around them.
\usepackage{wrapfig}
