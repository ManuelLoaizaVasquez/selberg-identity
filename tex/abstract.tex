\chapter*{Abstract}
One central theme of number theory is the distribution
of the prime numbers over the positive integers.
In one direction,
from the works of
Hadamard, de la Valle\'e Poussin and Newman,
we know that the PNT (Prime Number Theorem)
can be worked by complex analysis methods.
On another direction,
Selberg, Breusch, and Levinson proved the PNT
using elementary techniques in the sense that
it uses only real analysis.
Less than a decade ago,
Choudhary has strengthened Levinson's proof.
All the elementary proofs mentioned above
derive the PNT using Selberg's identity.

In this thesis,
we establish another proof for the Selberg's identity
simpler than Choudhary's in several respects
refining the works discussed earlier.
We also present a linear time algorithm for
estimating a formula derived from Selberg's identity.

\bigskip
\textbf{Keywords:}
Selberg's identity,
analytic number theory,
algorithmic number theory.

\bigskip
\textbf{2020 Mathematics Subject Classification:}
11A25,
11A41,
11Y16.
