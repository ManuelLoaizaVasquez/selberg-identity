\chapter*{Abstract}
One central theme of Number Theory is the distribution of
the prime numbers over the positive integers.
In one direction,
from the classic works of
Hadamard, Valle-Poussin and Newman,
we know that the PNT (Prime Number Theorem) holds
from Complex Analysis.
In another direction,
Selberg, Erdős and Levinson
have proved the PNT
using elementary techniques in the sense that
it uses only Real Analysis.
Less than a decade ago,
Choudhary has elementary proved the PNT
by strengthening Levinson's proof.

In this thesis,
we establish new proofs for
the Selberg's Identity
and the PNT
refining the works mentioned above
and develop efficient algorithms to estimate numeric results.

\begin{itemize}
    \item
    \textbf{New elementary proof of the Prime Number Theorem.}
    Following Selberg and Choudhary's frameworks,
    we prove the PNT without using
    Shapiro's Tauberian Theorems.

    \item
    \textbf{Sublinear Polylogarithmic Time Complexity Algorithms.}
    We develop algorithms in worst-case
    \(\wt{O}(n^{3/4})\), \(\wt{O}(n^{2/3})\) and \(O(n)\)
    time and implement them in C++
    for computing intermediate results
    and doing some numerical experimentation.
\end{itemize}

\bigskip
\textbf{Key Words:}
Prime Number Theorem,
Selberg's Identity,
Sieve Theory,
Analytic Number Theory,
Algorithmic Number Theory.

\bigskip
\textbf{2020 Mathematics Subject Classification:}
11A25,
11A41,
11Y11,
11Y16,
11Y60
11Y70.

\pagenumbering{roman}
\setcounter{page}{4}
