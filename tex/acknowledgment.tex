\chapter*{Agradecimientos}
\thispagestyle{empty}

\vspace{-0.5cm}

Primero, quiero agradecer profundamente a mi asesor Alfredo Poirier
por su guía y comprensión durante mi carrera universitaria.
Alfredo es un asesor fantástico:
un oráculo de perspicaces observaciones,
balanceado con humor
y anécdotas sobre su experiencia
con su asesor John Milnor
o su tiempo en el MSRI en UC Berkeley.

Aún recuerdo hace unos años atrás,
cuando cursaba el cuarto ciclo,
haber aprendido un poco sobre Teoría de Números
para los regionales de ACM ICPC
y divertirme con Alfredo
leyendo elegantes artículos
desarrollando aproximadamente el \(60\%\) de este trabajo,
el cual lo presentamos en el \(2020\)
y ganamos el
Primer Concurso de Iniciación a la Investigación en Estudios Generales Ciencias.
Jemisson y yo comenzamos a trabajar con Alfredo
tras tomar el curso Tópicos de Análisis al inicio de la pandemia
luego de una charla conjunta en la que nos escogió como asesorados:

\begin{displayquote}
La idea es salir de requisitos formales lo antes posible
y seguir avanzando en sus carreras de investigadores.
Las tesis deben tomarse como un juego,
y como tal,
las deben dar por finiquitadas lo antes posible.
\end{displayquote}

Desafortunadamente,
no seguí la línea de investigación
que Alfredo visionó
puesto que la pandemia
hizo florecer en mí
la adicción y el entusiasmo de construir
los bloques fundamentales de nuestras aplicaciones,
dedicándome profesionalmente
a la Ingeniería de Software
en paralelo a la carga académica
los dos años y medio posteriores,
motivo por el cual el \(40\%\) restante de esta tesis
está siendo concluido durante este décimo ciclo universitario.

Quiero agradecer a mis amistades a lo largo de los años:
Marcelo Gallardo,
Jemisson Coronel,
Eduardo Llamoca,
Diego Hurtado de Mendoza
y
Hans Acha.
Tengo la suerte de haber trabajado con estas personas supertalentosas.

Finalmente, quiero agradecer a mi familia por su apoyo constante e incondicional.
