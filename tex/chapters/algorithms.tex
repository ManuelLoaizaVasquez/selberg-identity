\pretolerance=20000
\tolerance=30000
\setlength{\headheight}{14.61858pt}
\selectlanguage{spanish}
\pagestyle{fancy}

\chapter[Experimentaci\'on num\'erica]
{Experimentaci\'on num\'erica}
\label{ch:algorithms}

\vspace*{1cm}

\section{Algoritmos}
En esta secci\'on introduciremos a detalle nuestro algoritmo
para estimar la f\'ormula asint\'otica de Selberg del \cref{ch:selberg}.
Primero, realicemos ciertas manipulaciones a la f\'ormula demostrada
\[
  \frac{\sum_{p \leq x} \ln^2 p + \sum_{pq \leq x} \ln p \ln q - 2x\ln x}{x} = O(1).
\]

Por definici\'on, esto significa que existe
un momento \(n_0 > 0\)
y una constante \(c > 0\)
tales que
\[
  K(x)
  = \abs{\frac{\sum_{p \leq x} \ln^2 p
  + \sum_{pq \leq x} \ln p \ln q
  - 2x\ln x}{x}}
  \leq c
\]
para todo \(x \geq n_0\).
Convenientemente hemos definido \(K\)
para poder estimar asint\'oticamente la identidad de Selberg.

Ahora presentaremos como computar \(K\).

Primero precomputaremos los n\'umeros primos en el rango de inter\'es
para poder utilizarlos en ambas sumatorias.

\begin{algorithm}[H]
    \SetAlgoLined
    \DontPrintSemicolon
    \Fn{\ObtenerPrimos{\(x\)}} {
        Sea \(M\brackets{0 \dots x}\) un nuevo arreglo est\'atico\;
        Inicializamos \(M\brackets{2 \dots x}\) con \textsc{Verdadero}\;
        \(M\brackets{0} = M\brackets{1} = \) \textsc{Falso}\;
        Sea \(P\) un nuevo arreglo din\'amico\;
        \For {\(n = 2\) \KwTo \(x\)} {
            \If {\(M\brackets{n}\)} {
                \textsc{Insert}\(\parentheses{P, n}\)\;
            }
            \For {\(p \in P\) and \(np \leq x\)} {
                \(M[np]\) = \textsc{Falso}\;
                \If {\(p \, \mid \, n\)} {
                    \textbf{break}\;
                }
            }
        }
        \KwRet{\(P\)}
    }
    \caption{Retorna un arreglo din\'amico con todos los n\'umeros primos menores o iguales a \(x\).}
\end{algorithm}

\begin{lemma}
    \label{lem:sieve-correctness}
    El algoritmo \texttt{ObtenerPrimos}
    computa correctamente los n\'umeros primos
    menores o iguales a \(x\).
\end{lemma}

\begin{proof}
    Definamos el arreglo est\'atico
    \(M\brackets{0 \dots x}\) en la l\'inea \(2\)
    donde \(M[i]\) denota si \(i\) es un n\'umero primo
    y el arreglo din\'amico \(P\) en la l\'inea \(5\)
    el cual contendr\'a todos los n\'umeros primos buscados.
    La l\'inea \(4\) inicializa \(M[0]\) y \(M[1]\) en \textsc{Falso}
    y el resto es inicializado como \textsc{Verdadero} en la l\'inea \(3\).
    Sea \(n\) el n\'umero que estamos analizando en las l\'ineas \(6-16\).
    Supongamos que \(n\) es un n\'umero primo.
    Luego, las l\'ineas \(7-9\) siempre insertar\'an
    a \(n\) en \(P\) puesto que la \'unica manera que \(M[n]\) sea \textsc{Falso}
    es que alg\'un primo \(p \in P\) lo haya etiquetado de esta manera
    en la l\'inea \(11\), lo cual es imposible pues \(n < pn\).
    Por el teorema fundamental de la aritm\'etica
    podemos expresar de manera \'unica
    cualquier n\'umero natural mayor que \(1\) como
    producto de potencias de n\'umeros primos.
    Sin p\'erdida de generalidad, expresemos
    \(n = p_1^{\alpha_1} \dots p_k^{\alpha_k}\)
    con los factores primos \(p_1 \leq \dots \leq p_k\) ordenados crecientemente.
    Ahora supongamos que \(n\) es compuesto y escribamos
    \(n = p_1 m\) con \(m = p_1^{\alpha_1 - 1} \dots p_k^{\alpha_k}\).
    Afirmamos que \(M[n]\) es \textsc{Falso} al llegar a la l\'inea \(7\)
    y, por ende, no es insertado en \(P\).
    Necesariamente \(1 < m < n\) pues \(n\) es compuesto.
    Como \(p_1\) es menor o igual que
    todos los primos que dividen a \(m\),
    la l\'inea \(11\) asigna a \(M[m p_1]\)
    el valor \textsc{Falso}
    cuando el bucle de las l\'ineas \(6-16\)
    est\'a analizando a \(m\).
\end{proof}

\begin{lemma}
    \label{lem:sieve-time}
    El tiempo de ejecuci\'on del algoritmo \texttt{ObtenerPrimos}
    para computar los n\'umeros primos menores o iguales a \(x\) es \(O(x)\).
\end{lemma}

\begin{proof}
    En la prueba del \cref{lem:sieve-correctness} observamos que
    cada n\'umero compuesto es etiquetado como \textsc{Falso}
    una \'unica vez puesto que determin\'isticamente sabemos qu\'e primo
    y qu\'e n\'umero su producto hizo que cambie el valor de \(M\) en dicho compuesto.
    En la l\'inea \(12\) nos aseguramos de que los m\'ultiplos a analizar
    sean originados por el menor primo que divide a este n\'umero compuesto,
    pues ni bien deje de serlo,
    la l\'inea \(13\) evita
    que sigamos analizando m\'as m\'ultiplos innecesariamente.
\end{proof}

\begin{theorem}
    \label{thm:sieve}
    Sea \(x\) un n\'umero entero positivo.
    El algoritmo \texttt{ObtenerPrimos}
    computa correctamente los n\'umeros primos
    menores o iguales a \(x\) en tiempo y espacio \(O(x)\).
\end{theorem}

\begin{proof}
    Inmediato del \cref{lem:sieve-correctness} y el \cref{lem:sieve-time}.
\end{proof}

Vamos preparando el clima para computar \(\sum_{pq \leq x} \ln p \ln q\).
Si fijamos \(p\) y conocemos todos los primos menores o iguales a \(x\),
me gustar\'ia saber qu\'e primos \(q\) mayores o iguales a \(p\) aportan a esta sumatoria.
El siguiente algoritmo nos permitir\'a obtener la menor cota superior
para este subconjunto.

\begin{algorithm}[H]
    \SetAlgoLined
    \DontPrintSemicolon
    \Fn{\ObtenerMaximoPrimo{\(p, x, P\)}} {
        \(l = 0\)\;
        \(r = P.\,length - 1\)\;
        \If {\(p \cdot P\brackets{r} \leq x\)} {
            \KwRet{\(P\brackets{r}\)}
        }
        \If {\(p^2 > x\)} {
            \KwRet{\(-1\)}
        }
        \While {\(r - l > 1\)} {
            \(m = \floor{\parentheses{l + r} / 2}\)\;
            \eIf {\(p \cdot P\brackets{m} \leq x\)} {
                \(l = m\)\;
            }
            {
                \(r = m\)\;
            }
        }
        \KwRet{\(P\brackets{l}\)}
    }
    \caption{Retorna el m\'aximo primo \(q\) mayor o igual a \(p\) tal que \(pq \leq x\).}
\end{algorithm}

\begin{theorem}
    \label{thm:binary-search}
    Sea \(x\) un n\'umero entero positivo,
    \(p\) un n\'umero primo menor o igual a \(x\) y
    \(P\) un arreglo din\'amico con los n\'umeros primos menores o iguales a \(x\) ordenados crecientemente.
    El algoritmo \texttt{ObtenerM\'aximoPrimo}
    obtiene el m\'aximo primo \(q\)
    mayor o igual a \(p\) tal que
    \(pq \leq x\)
    o \(-1\) en caso no exista
    en tiempo \(O(\log_2 \pi(x))\).
\end{theorem}

\begin{proof}
    Una precondici\'on para la l\'inea \(4\) es que
    \(P[l]\) y \(P[r]\) son el menor y mayor n\'umero primo
    menores o iguales a \(x\), respectivamente.
    En las l\'ineas \(4-6\), si \(p\) multiplicado por el mayor n\'umero primo menor o igual a \(x\)
    sigue siendo menor o igual a \(x\), este ser\'ia el primo buscado.
    Por otro lado, si \(p^2 > x\) tenemos \(pq \geq p^2 > x\) para \(q \geq p\)
    por lo que no existir\'ia soluci\'on.
    As\'i, las l\'ineas \(7-9\) se aseguran de retornar \(-1\) frente a esta situaci\'on.
    Caso contrario, notamos que si \(p \cdot P[i] \leq x\)
    entonces \(p \cdot P[i - 1] \leq x\) para \(i > 0\).
    Luego construimos la invariante \(p \cdot P[l] \leq x\) y \(p \cdot P[r] > x\)
    y la preservamos a lo largo del bucle de las l\'ineas \(10-17\),
    el cual se detiene cuando \(l\) y \(r\) son consecutivos.
    Como \(P\) tiene \(\pi(x)\) elementos y en cada paso
    reducimos el espacio de b\'usqueda de \(r - l + 1\) elementos
    a \(\floor{(r - l + 1) / 2} + O(1)\) elementos,
    la complejidad en tiempo ser\'a \(O(\log_2 \pi(x))\).
\end{proof}

Supongamos que tenemos un arreglo \(L[0 \dots x]\) donde
\(L[n] = \sum_{p \leq n} \ln p\).
Luego podemos calcular \(\sum_{l \leq p \leq r} \ln p = L[r] - L[l - 1]\) para \(l \geq 1\)
en \(O(1)\).
La meta del siguiente algoritmo es computar \(L[0 \dots x]\)
y de pasada a \(\sum_{p \leq x} \ln^2 p\).

\begin{algorithm}[H]
    \SetAlgoLined
    \DontPrintSemicolon
    \Fn{\SumarUno{\(x, P\)}} {
        Sea \(L\brackets{0 \dots x}\) un nuevos arreglo est\'atico\;
        \(L\brackets{0} = 0\)\;
        \(s = 0\)\;
        \(i = 0\)\;
        \For {\(n = 1\) \KwTo \(x\)} {
            \(L\brackets{n} = L\brackets{n - 1}\)\;
            \If {\(i < P.\,length\) and \(P\brackets{i} == n\)} {
                \(L\brackets{n} = L\brackets{n} + \ln n\)\;
                \(s = s + \ln^2 n\)\;
                \(i = i + 1\)\;
            }
        }
        \KwRet{\(s\) y \(L\)}
    }
    \caption{Retorna \(\sum_{p \leq x} \ln^2 p\) y las sumas parciales de \(\sum_{p \leq x} \ln p\).}
\end{algorithm}

\begin{theorem}
    \label{thm:sum-1}
    Sea \(x\) un n\'umero entero positivo y
    \(P\) un arreglo din\'amico con los n\'umeros primos menores o iguales a \(x\) ordenados crecientemente.
    El algoritmo \texttt{Sumar\textsubscript{1}} computa \(\sum_{p \leq x} \ln^2 p\) y \(L[0 \dots x]\)
    en tiempo \(O(x)\).
\end{theorem}

\begin{proof}
    Trivialmente el tiempo de ejecuci\'on del algoritmo es \(O(x)\) pues
    cada iteraci\'on del bucle en las l\'ineas \(6-13\) toma \(O(1)\).
    Adem\'as, es t\'acito que el aporte de \(\ln^2 n\) a \(s\) y de \(\ln n\) a \(L[n]\)
    se da cuando \(n\) es primo.
\end{proof}

Ya hemos preparado el clima
y dispuesto los tecnicismos necesarios para
computar el plato de fondo.
Sea \(x\) un n\'umero entero positivo, expresamos
\[
    \sum_{pq \leq x} \ln p \ln q
    = \sum_{p \leq x} \sum_{p \leq q \leq r} \ln p \ln q
    = \sum_{p \leq x} \ln p \sum_{p \leq q \leq r} \ln q
    = \sum_{p \leq x} \ln p \cdot \parentheses{L[r] - L[p - 1]}
\]
donde \(r\) es el m\'aximo primo mayor o igual a \(p\) tal que \(pr \leq x\).

\begin{algorithm}[H]
    \SetAlgoLined
    \DontPrintSemicolon
    \Fn{\SumarDos{\(x, L, P\)}} {
        \(s = 0\)\;
        \For {\(p \in P\)} {
            \(r =\,\)\texttt{ObtenerM\'aximoPrimo}\(\parentheses{p, x, P}\)\;
            \If {\(r == -1\)} {
                \textbf{break}\;
            }
            \(s = s + \ln p \cdot \parentheses{L\brackets{r} - L\brackets{p - 1}}\)\;
        }
        \KwRet{\(s\)}
    }
    \caption{Retorna \(\sum_{pq \leq x} \ln p \ln q\).}
\end{algorithm}

\begin{theorem}
    \label{thm:sum-2}
    Sea \(x\) un n\'umero entero positivo,
    \(P\) un arreglo din\'amico con los n\'umeros primos menores o iguales a \(x\) ordenados crecientemente y
    \(L[0 \dots x]\) un arreglo en donde \(L[n] = \sum_{p \leq n} \ln p\).
    El algoritmo \texttt{Sumar}\textsubscript{2} computa \(\sum_{pq \leq x} \ln p \ln q\) en \(O(x)\).
\end{theorem}

\begin{proof}
    El bucle de las l\'ineas \(3-9\) realizar\'a a lo m\'as \(\pi(x)\) iteraciones.
    En cada iteraci\'on, la l\'inea \(4\) toma \(O(\log_2\pi(x))\) en tiempo
    y el resto de l\'ineas toma \(O(1)\) por lo que el tiempo de ejecuci\'on ser\'ia
    \(O\parentheses{\pi(x) \log_2\pi(x)}\).
    Rosser y Schoenfeld \cite[corolario 1]{Rosser1962} probaron que se cumple
    \(\pi(x) < cx / \ln x\) para \(x > 1\) y \(c = 1.25506\).
    Tras un mero manejo simb\'olico
    \begin{align*}
        \pi(x) \log_2 \pi(x) &\leq \frac{cx}{\ln x} \log_2 \frac{cx}{\ln x} \\
        &= \frac{cx}{\ln x} \parentheses{\log_2 c + \log_2 x - \log_2 \ln x} \\
        &= \parentheses{c \log_2 c}\frac{x}{\ln x} + \parentheses{c \log_2 e}x - c\parentheses{\frac{x \log_2 \ln x}{\ln x}} \\
        &= O(x),
    \end{align*}
    nuestra labor es recompensada inmediatamente.
\end{proof}

El siguiente algoritmo en cierto sentido resume nuestro trabajo hasta el momento.

\begin{algorithm}[H]
    \SetAlgoLined
    \DontPrintSemicolon
    \Fn{\K{\(x\)}} {
        \(P = \,\)\texttt{ObtenerPrimos}\((x)\)\;
        \(s \, , L = \,\)\texttt{Sumar}\(_1(x, P)\)\;
        \(t = \,\)\texttt{Sumar}\(_2(x, L, P)\)\;
        \KwRet{\(\abs{s + t - 2 x \ln x}/x\)}
    }
    \caption{Computa \(K(x)\).}
\end{algorithm}

\begin{theorem}
    Sea \(x\) un n\'umero entero positivo.
    El algoritmo \texttt{K} computa \(K(x)\) en \(O(x)\).
\end{theorem}

\begin{proof}
    Esto es fruto de el \cref{thm:sieve}, el \cref{thm:sum-1} y el \cref{thm:sum-2}.
\end{proof}

\section{Estimados}

Realizamos nuestros experimentos en una
MacBook Pro \(15-\)inch \(2017\)
corriendo macOS Ventura \(13.4\)
con \(16\)GB de RAM y
un CPU Quad-Core Intel Core i7
con m\'axima frecuencia de reloj de \(2.9\)GHz.
El procesador contiene \(4\) n\'ucleos
y cada n\'ucleo puede aprovechar \(2\) hilos
con hyper-threading technology habilitada.
Usamos el lenguaje de programaci\'on C++ para implementar nuestros algoritmos,
utilizando el est\'andar C++17.
Los n\'umeros de \(64\) bits fueron emulados usando los tipos de datos
\texttt{long long} y \texttt{long double} del compilador
Apple clang version 14.0.3 (clang-1403.0.22.14.1).

\begin{table}[htbp]
    \centering
    \caption{Resultados y tiempos de computaci\'on en la MacBook Pro.}
    \begin{tabular}{cccc}
      \toprule
      \textbf{\(x\)} & \textbf{\(K(x)\)} & \textbf{Tiempo de computaci\'on (ms)} \\
      \midrule
      \(10^1\) & 3.4422963847 & 0 \\
      \(10^2\) & 5.1926865457 & 0 \\
      \(10^3\) & 6.3173110617 & 0 \\
      \(10^4\) & 7.4463872101 & 2 \\
      \(10^5\) & 8.5931825594 & 18 \\
      \(10^6\) & 9.7471335703 & 156 \\
      \(10^7\) & 10.8942869828 & 1406 \\
      \(10^8\) & 12.0436377380 & 16150 \\
      \(10^9\) & 13.1943961187 & 201913 \\
      \bottomrule
    \end{tabular}
\end{table}

El c\'odigo fuente para replicar los resultados
está disponible en el \cref{ch:src-code}.
Además, puedes encontrarlo en GitHub en
\goto{https://github.com/ManuelLoaizaVasquez/selberg-identity/tree/main/code/}.