\pretolerance=20000
\tolerance=30000
\setlength{\headheight}{14.61858pt}
\selectlanguage{spanish}
\pagestyle{fancy}

\chapter[Algoritmos]
{Algoritmos}
\label{Chapter04}

\vspace*{1cm}

\section{Algoritmo Principal}
Adicional a la prueba del Teorema \ref{the22},
hemos dise\~nado unos algoritmos para poder obtener estimados
con la f\'ormula asint\'otica de Selberg.
Primero, realicemos ciertas manipulaciones a la f\'ormula demostrada
\begin{gather}
  \sum_{p \leq x} \ln^2 p + \sum_{pq \leq x} \ln p \ln q = 2x\ln x + O(x) \\
  \sum_{p \leq x} \ln^2 p + \sum_{pq \leq x} \ln p \ln q - 2x\ln x = O(x) \\
  \frac{\sum_{p \leq x} \ln^2 p + \sum_{pq \leq x} \ln p \ln q - 2x\ln x}{x} = O(1).
\end{gather}

Por definici\'on, esto significa que existe un momento \(n_0 > 0\)
y una constante \(c > 0\) a partir del cual
\begin{align*}
  \abs{\frac{\sum_{p \leq x} \ln^2 p + \sum_{pq \leq x} \ln p \ln q - 2x\ln x}{x}} \leq c
\end{align*}
para todo \(x \geq n_0\).

Asumamos que tenemos un n\'umero \(x\) y queremos calcular
\begin{align*}
  \abs{\frac{\sum_{p \leq x} \ln^2 p + \sum_{pq \leq x} \ln p \ln q - 2x\ln x}{x}}.
\end{align*}

La estrategia que utilizaremos consiste en lo siguiente:
Primero precalcularemos los n\'umeros primos
en el rango \(\brackets{1 \dots x}\).
Como solamente manipularemos logaritmos de n\'umeros primos,
aprovechar\'e de esto para poder precalcular las sumatorias y
hallar sumas en rangos en \(O(1)\) como diferencia de sumas.
El contenedor \(\log\) almacenar\'a lo siguiente:
\[
  \log[p] =
  \begin{cases}
    \ln p &\quad\text{si } p \text{ es primo,}\\
    0 &\quad\text{en otro caso.}
  \end{cases}
\]

As\'i, \(\sum_{i = l}^r \log[i]\)
solo tendr\'a la sumatoria de los logaritmos naturales de
los n\'umeros primos en el intervalo \(\brackets{l, r}\).

El contenedor es\_primo almacenar\'a lo siguiente:
\[
  \text{es\_primo[$p$]} =
  \begin{cases}
    \text{verdadero} &\quad\text{si }$p$\text{ es primo,}\\
    \text{falso} &\quad\text{en otro caso.}
  \end{cases}
\]

Los contenedores suma\_\(\log\) y suma\_\(\log^2\) almacenar\'an
\[
  \text{suma\_\(\log[x]\)} = \sum_{i = 0}^x \log[i]
\]
y
\[
  \text{suma\_\(\log^2[x]\)} = \sum_{i = 0}^x \log^2[i].
\]

El primer preprocesamiento que realizaremos ser\'a llamar al m\'etodo CribaLineal.
Este m\'etodo reibir\'a como par\'ametro al n\'umero n. 

\begin{algorithm}[H]
\SetAlgoLined
\DontPrintSemicolon
\KwData{es\_primo, primos, $n$.}
\KwResult{N\'umeros primos en el rango $[1 \dots n]$ guardados en primos.}
\Begin{
    es\_primo[$0$] $\leftarrow$ falso\;
    es\_primo[$1$] $\leftarrow$ falso\;
    \For{$i \leftarrow 2$ \KwTo $n$} {
        es\_primo[$i$] $\leftarrow$ verdadero\;
    }
    primos $\leftarrow$ $\emptyset$\;
    \For{$i \leftarrow 2$ \KwTo $n$} {
        \If{es\_primo[$i$]} {
            primos $\leftarrow$ primos $\cup\;\{i\}$\;
        }
        \For{$p \in$ primos \textbf{and} $i \cdot p \leq n$}{
            es\_primo[$i \cdot p$] $\leftarrow$ falso\;
            \If{$i \equiv 0 \mod{p}$}{
                \textbf{break}\;
            }
        }
    }
}
\caption{CribaLineal\label{CL}}
\end{algorithm}

\begin{lemma}
Sea $n$ el n\'umero que representa el extremo derecho del intervalo $[1 \dots n]$
en el cual queremos hallar todos los n\'umeros primos mediante la ejecuci\'on de CribaLineal. Luego el tiempo de ejecuci\'on es $O(n)$.
\end{lemma}

\begin{proof}
\end{proof}

\begin{algorithm}[H]
\SetAlgoLined
\DontPrintSemicolon
\KwData{$i, x,$ primos}
\KwResult{Posici\'on del mayor n\'umero primo $q$ tal que $pq \leq x$.
En caso no exista, se retornar\'a $-1$.}
\Begin{
    $p \leftarrow$ primos[$i$]\;
    $l \leftarrow i$\;
    $r \leftarrow |$primos$| - 1$\;
    \If{$p \cdot primos[r] \leq x$}{
        \textbf{return} r\;
    }
    \If{$p^2 > x$}{
        \textbf{return} $-1$\;
    }
    \While{$r - l > 1$}{
        $m \leftarrow \floor*{\frac{l + r}{2}}$\;
        \eIf{$p \cdot primos[m] \leq x$} {
            $l \leftarrow m$\;
        } {
            $r \leftarrow m$\;
        }
    }
    \textbf{return} $l$\;
}
\caption{Buscar\'UltimaPosici\'on\label{BUP}}
\end{algorithm}

\begin{lemma}
Sea $i$ el \'indice que representa al $i$-\'esimo n\'umero primo
y $p$ este $i$-\'esimo n\'umero primo.
Sea $x$ el n\'umero que representa la cota superior para el producto de $p$ con otro n\'umero primo $q$ tal que $pq \leq x$ y $q \geq p$.
Obtendremos la posici\'on del mayor n\'umero primo $q$ que cumpla con lo anterior o
$-1$ en caso este n\'umero primo no exista
mediante la ejecuci\'on de Buscar\'UltimaPosici\'on.
Luego el tiempo de ejecuci\'on es $O(\log_2 n)$.
\end{lemma}

\begin{proof}
Obtener el valor del $i$-\'esimo n\'umero primo e inicializar
nuestros extremos de los intervalos para realizar la b\'usqueda binaria en las l\'ineas
$2 - 4$ toma $O(1)$ en tiempo.
Nuestro objetivo es encontrar la \'ultima posici\'on $j$ en la cual
el $j$-\'esimo n\'umero primo multiplicado por $p$ sea menor o igual a $x$ con $i \leq j$. Antes de analizar la invariante, quit\'emonos de encima los casos borde.
El primer caso borde es en el cual el menor elemento en nuestro rango no cumple la condici\'on,
en este caso retornamos $-1$, pues para todo $q > p$ tenemos que $pq > p^2 > x$,
por lo que no existir\'a par que satisfaga la condici\'on.
El segundo caso borde es cuando el n\'umero de la \'ultima posici\'on cumple.
En este caso retornamos de inmediato esta \'ultima posici\'on,
pues para cualquier $q < p_r$ tenemos $pq < p \cdot p_r \leq x$,
por lo cual todos los dem\'as primos en el rango tambi\'en cumplir\'ian.
Analizar ambos casos nos tomar\'ia $O(1)$ en tiempo.
Al entrar al bucle en las l\'ineas $11 - 18$ se satisface $i = l < r = \pi(x) - 1$ y adem\'as
$p\;\cdot$ primos[$l$] $\leq x$ y $p\;\cdot$ primos[$r$] $> x$.
Mientras que $l$ y $r$ disten al menos 2, computaremos $m = \floor*{\frac{l + r}{2}}$
y podemos distinguir dos casos:
el primer caso es cuando se cumple $p\;\cdot$ primos[$m$] $\leq x$, $l$ cambiar\'ia
su valor a $m$ y, debido a que la diferencia entre $l$ y $r$ era mayor a 1, entonces
$l' = m < r$ y la condici\'on $p\;\cdot$ primos[$l'$] $\leq x$ y $p\;\cdot$ primos[$r$] $> x$ se sigue cumpliendo.
En el segundo caso tenemos $p\;\cdot$ primos[$m$] $> x$, aqu\'i $r' = m$ y cuando $r' > l$ la condici\'on se cumple de manera an\'aloga al caso anterior,
pero cuando $r' = l$, tenemos que en la siguiente iteraci\'on del bucle, al ya no distar $2$, saldr\'iamos del bucle
y necesariamente la respuesta se encuentra en nuestro extremo izquierdo $l$.
Ahora que ya sabemos que siempre terminamos obteniendo el \'ultimo elemento que cumple la condici\'on, analicemos el
orden de complejidad de este algoritmo. Sea $T: \BN \to \BN$ la funci\'on que cuenta la cantidad de operaciones
que realiza el bucle con $T(n) = 4 + T(\ceiling{\frac{n}{2}})$,
puesto que en el peor de los casos nos quedamos con la mitad m\'as grande;
no obstante, esto lo podemos expresar como $T(n) = O(1) + T(\frac{n}{2})$ seg\'un \cite{CLRS}
y utilizando inducci\'on sobre esta \'ultima conseguimos $T(n) = O(\log_2 n)$.
En el peor de los casos, $n = \pi(x)$, por lo que la complejidad total del algoritmo ser\'ia
$O(\log_2 \pi(x))$, el cual es a su vez $O(\log_2 x)$ por definici\'on.
\end{proof}

\begin{algorithm}[H]
\SetAlgoLined
\DontPrintSemicolon
\KwData{primos, suma\_log, $x$.}
\KwResult{$\sum_{pq \leq x} \ln p \ln q$.}
\Begin{
    $suma \leftarrow 0$\;
    \For{$pos_p \leftarrow 0$ \KwTo $|$primos$| - 1$}{
        $pos_q \leftarrow$ Buscar\'UltimaPosici\'on($pos_p$, $x$)\;
        \If{$pos_q = -1$}{
            \textbf{break}\;
        }
        $p \leftarrow$ primos$[pos_p]$\;
        $q \leftarrow$ primos$[pos_q]$\;
        $suma \leftarrow suma + \ln p \cdot ($suma\_log$[q] - $suma\_log$[p - 1])$\;
    }
    \textbf{return} $suma$\;
}
\caption{CalcularSuma\label{CS}}
\end{algorithm}

\begin{lemma}
Sea $x$ el n\'umero que representa la cota superior para el producto de dos n\'umeros primos
$p$ y $q$ en $\sum_{pq \leq x} \ln p \ln q$.
Si obtenemos el valor de esta sumatoria mediante la ejecuci\'on de CalcularSuma, 
el tiempo de ejecuci\'on es $O(x)$.
\end{lemma}

\begin{proof}
Tenemos los n\'umeros primos en el rango $[1 \dots x]$ en el contenedor ordenado $primos$,
entonces por definici\'on $\pi(x)=|primos| $.
Inicializar la suma en la l\'inea $2$ toma $O(1)$ en tiempo.
El bucle en las l\'ineas $3-11$ ser\'a ejecutado $O(\pi(x))$ veces.
La funci\'on Buscar\'UltimaPosici\'on en la l\'inea $4$ es llamada exactamente una vez para cada primo $p \in primos$ y
toma $O(\log_2 x)$ en tiempo de acuerdo con el lema 17, 
las dem\'as operaciones dentro del bucle toman $O(1)$ en tiempo.
De esta manera, el tiempo de ejecuci\'on del algoritmo es $O(\pi(x)\log_2 x)$.
Asimismo, Rosser y Barkley \cite[teorema 2 y corolario 1]{Chebyshev} probaron lo siguiente:
\begin{align}
\pi(x) \leq 1.25506\;\frac{x}{\ln x}. %%%%\;x > 1.
\end{align}
Utilizamos esto para tener un mejor estimado cual 
\begin{align}
    \pi(x)\log_2 x &\leq 1.25506\;\frac{x}{\ln x} \log_2 x\\
    &= 1.25506\;\log_2 e \frac{x}{\log_2 e \cdot \log_e x} \log_2 x\\
    &= 1.25506\;\log_2 e \frac{x}{\log_2 x} \log_2 x\\
    &= (1.25506\;\log_2 e) x\\
\end{align}
y concluimos que el tiempo de ejecuci\'on del algoritmo es $O(x)$.
\end{proof}

\begin{algorithm}[H]
\SetAlgoLined
\DontPrintSemicolon
\KwData{$x.$}
\KwResult{$\left|\frac{\sum_{p \leq x} \ln^2 p + \sum_{pq \leq x} \ln p \ln q - 2x\ln x}{x}\right|.$}
\Begin{
    CribaLineal($x$)\;
    \For{$i \leftarrow 1$ \KwTo $x$}{
        suma\_log$[i] \leftarrow 0$\;
        suma\_$\log^2[i] \leftarrow 0$\;
        \eIf{$es\_primo[i]$}{
            $\log[i] \leftarrow \ln i$\;
        }{
            $\log[i] \leftarrow 0$\;
        }
    }
    \For{$i \leftarrow 2$ \KwTo $x$}{
        suma\_log[$i] \leftarrow$ suma\_$\log[i - 1] + \log[i]$\;
        suma\_$\log^2[i] \leftarrow$ suma\_$\log^2[i - 1] + (\log[i])^2$\;
    }
    \textbf{return} $\frac{|\text{suma}\_\log^2[x] + \text{CalcularSuma}(x) - 2x\ln x|}{x}$\;
}
\caption{EstimarConstante}
\end{algorithm}

\begin{theorem}
Con $x$ el n\'umero natural para el cual se quiere determinar
$$\frac{|\sum_{p \leq x} \ln^2 p + \sum_{pq \leq x} \ln p \ln q - 2x\ln x|}{x}$$
mediante la ejecuci\'on de EstimarConstante
resulta que el tiempo de ejecuci\'on es $O(x)$.
\end{theorem}

\begin{proof}
\end{proof}

\section{Estimados}

Sea $t$ la cantidad de casos de prueba y $n$ el m\'aximo valor que puede tomar $x$ en la expresi\'on que analizaremos. 
El siguiente programa en C++ determina el valor de 
$$\frac{|\sum_{p \leq x} \ln^2 p + \sum_{pq \leq x} \ln p \ln q - 2x\ln x|}{x}$$
para cada caso de prueba.
En particular, usaremos $t = 28$ y $n = 3 \cdot 10^7$.
La complejidad asint\'otica en tiempo del programa es $O(tn)$, la cual se ve reflejada en un tiempo de ejecuci\'on
de tan solo tres segundos.

\begin{center}
    \texttt{real	0m3,123s}\\
    \texttt{user	0m2,675s}\\
    \texttt{sys	0m0,436s}.
\end{center}

% \cppfile{formula_selberg.cpp}

\begin{table}
\centering
\caption{Resultado del programa en cada uno de los 28 casos de prueba.}
\begin{tabular}{@{} l *{5}{S[table-format=-1.7]} @{}} 
\toprule
{$x$} & 
{$\sum_{p \leq x} \ln^2 p + \sum_{pq \leq x} \ln p \ln q - 2x\ln x$} & 
{$\frac{|\sum_{p \leq x} \ln^2 p + \sum_{pq \leq x} \ln p \ln q - 2x\ln x|}{x}$}\\ % center-set header entries
\midrule
10 & -34.4229638465 & 3.4422963847\\
100 & -519.2686545658 & 5.1926865457\\
1000 & -6317.3110617078 & 6.3173110617\\
10000 & -74463.8721010727 & 7.4463872101\\
100000 & -859318.2559356594 & 8.5931825594\\
1000000 & -9747133.5703212193 & 9.7471335703\\
1500000 & -14918429.3651946987 & 9.9456195768\\
2000000 & -20177763.7803012519 & 10.0888818902\\
2500000 & -25505439.7770474999 & 10.2021759108\\
3000000 & -30875441.5873458827 & 10.2918138624\\
3500000 & -36291518.9878187147 & 10.3690054251\\
4000000 & -41746678.7374733434 & 10.4366696844\\
4500000 & -47221448.8244719136 & 10.4936552943\\
5000000 & -52725446.2863963772 & 10.5450892573\\
5500000 & -58263339.7575164592 & 10.5933345014\\
6000000 & -63830467.8915757891 & 10.6384113153\\
6500000 & -69414402.9962517989 & 10.6791389225\\
7000000 & -74998328.6200477667 & 10.7140469457\\
7500000 & -80619769.8168214446 & 10.7493026422\\
8000000 & -86252298.0080809856 & 10.7815372510\\
8500000 & -91907189.6865512790 & 10.8126105514\\
9000000 & -97554179.0078311940 & 10.8393532231\\
9500000 & -103244044.9516971762 & 10.8677942054\\
10000000 & -108942869.8283277971 & 10.8942869828\\
15000000 & -166436611.9307800680 & 11.0957741287\\
20000000 & -224778414.1259130784 & 11.2389207063\\
25000000 & -283766370.6499844969 & 11.3506548260\\
30000000 & -343248619.2823745428 & 11.4416206427\\
\bottomrule
\end{tabular}
\end{table}
