\pretolerance=20000
\tolerance=30000
\setlength{\headheight}{14.61858pt}
\selectlanguage{spanish}
\pagestyle{fancy}

\chapter[Introducci\'on]
{Introducci\'on}
\label{ch:introduction}

\vspace*{1cm}

El teorema del n\'umero primo ha sido extensivamente estudiado
\cite{
  Breusch1960,
  Choudhary2017,
  Diamond1982,
  Erdos1949,
  Levinson1969,
  Liu2022,
  Newman1980,
  Pan2023,
  Richter2021,
  Selberg1949,
  Shapiro1959}.
Este teorema afirma que la funci\'on \(\pi(x)\) se aproxima
asint\'oticamente a \(x / \ln x\).

En \(1852\), Chebyshev \cite{Chebyshev1852} acota \(\pi(x) / (x / \ln x)\)
y concluye que el l\'imite es igual a \(1\) en caso exista.
En \(1896\), Hadamard \cite{Hadamard1896} y de la Valle\'e Poussin
probaron el PNT v\'ia la funci\'on \(\zeta\) de Riemann.
En \(1949\), Selberg \cite{Selberg1949}
prob\'o este teorema elegantemente sin uso de an\'alisis complejo.
Un par de meses luego, Erdős \cite{Erdos1949}
prob\'o el teorema mediante el abuso de estimados tauberianos.
En \(1959\), Shapiro \cite{Shapiro1959}
prueba un par de teoremas tauberianos y equivalencias relacionadas a \cite{Erdos1949},
lo cual desemboc\'o tambi\'en en una nueva prueba del PNT.
En \(1969\), Levinson \cite{Levinson1969} se inspira de \cite{Selberg1949, Breusch1960}
para crear una demostraci\'on elemental
profundizando el estudio de la funci\'on resto.
En \(2017\), Choudhary \cite{Choudhary2017} prueba elementalmente el PNT
reemplazando ciertos resultados de \cite{Levinson1969}
por corolarios de \cite{Shapiro1959}.

\section{Nuestros resultados}

El prop\'osito de este trabajo es presentar una nueva prueba elemental del PNT
tras demostrar la f\'ormula de Selberg en todo rigor.

\bigskip

Empezamos enunciando la f\'ormula asint\'otica de Selberg. 
En todo lo que sigue los s\'imbolos \(p\) y \(q\) se referir\'an a
n\'umeros primos positivos.

\begin{theorem}
  [Identidad de Selberg]
  Para todo n\'umero real \(x\) mayor o igual a \(1\)
  se cumple la f\'ormula
  \[
    \sum_{p \leq x} \ln^2 p + \sum_{pq \leq x} \ln p \ln q = 2x\ln x + O(x).
  \]
\end{theorem}

Ahora enunciamos uno de los todopoderosos teoremas
de la teor\'ia anal\'itica de n\'umeros.
Considere \(\pi(x)\) la funci\'on que cuenta
la cantidad de n\'umeros primos menores o iguales a \(x\).

\begin{theorem}
  [Teorema del n\'umero primo]
  El l\'imite
  \(\displaystyle\lim_{x \to \infty} \frac{\pi(x)}{x / \ln x}\)
  existe y es igual a \(1\).
\end{theorem}

\section{Nuestras t\'ecnicas}
Para obtener nuestros resultados,
describiremos a continuaci\'on
c\'omo hemos hecho uso del an\'alisis real y
la teor\'ia anal\'itica de n\'umeros.

\textbf{Identidad de Selberg.}
\cite[secci\'on 2]{Selberg1949} nos provee esta identidad como
la herramienta m\'as poderosa para alcanzar el premio gordo.
Para demostrarla, usaremos como hoja de ruta \cite{Choudhary2017}
sin utilizar las f\'ormulas que derivan de \cite[teorema 3.17]{Choudhary2017}
ni los estimados tauberianos de Shapiro desarrollados en \cite[secci\'on 4]{Choudhary2017}.
En su reemplazo, utilizaremos nuestra creatividad y
ciertos resultados de \cite{Apostol1976, Diamond1982, TI1951}.

\textbf{Funci\'on resto.}
Nuestro an\'alisis de la funci\'on resto es una modificaci\'on de
las propiedades estudiadas en \cite[secci\'on 3]{Selberg1949} usando la desigualdad
\[
  \abs{R(x)} \leq \frac{2}{\ln^2 x}
  \sum_{n \leq x} \frac{\ln n}{n} \abs{R \parentheses{\frac{x}{n}}}
  + O\parentheses{\frac{x}{\ln x}}
\]
en reemplazo de
\[
  \abs{R(x)} \leq \frac{1}{\ln x}
  \sum_{n \leq x} \abs{R \parentheses{\frac{x}{n}}}
  + O\parentheses{x \frac{\ln \ln x}{\ln x}}.
\]

\section{Notaci\'on}
Emplearemos \(f(x) = O(g(x))\) en vez de \(f \in O(g)\)
a pesar de que no se trate de una igualdad de conjuntos sino
pertenencia de una funci\'on a una clase determinada de funciones.
De la misma manera trataremos la aritm\'etica entre familias de funciones
con notaci\'on \(O\).
Los s\'imbolos \(p\) y \(q\), en caso de no especificarse,
har\'an referencia a n\'umeros primos positivos.

% \section{Organizaci\'on}
% En la secci\'on 2 presentaremos los teoremas y definiciones que no probaremos
% pero  son lugar com\'un en \'area,
% utilizaremos como referencia
% \cite{Apostol1976}.
% En la secci\'on 3 realizaremos la demostraci\'on de la identidad de Selberg
% tras la prueba de ciertos lemas intermedios.
