\pretolerance=20000
\tolerance=30000
\setlength{\headheight}{14.61858pt}
\selectlanguage{spanish}
\pagestyle{fancy}

\chapter[Introducci\'on]
{Introducci\'on}
\label{Chapter01}

\vspace*{1cm}

Riemann, Erdos, Selberg, Newman, Tao y otros matem\'aticos han demostrado el
teorema del n\'umero primo de distintas maneras as\'i como Chebyshev y Euler
lograron resultados parciales.
Todos aportaron teoremas y t\'ecnicas que han
logrado el desarrollo de nuevas teor\'ias, as\'i como la soluci\'on
de conjeturas y propuestas de hip\'otesis a\'un sin una demostraci\'on.
Adem\'as, desde hace miles de a\~nos Euclides y Erat\'ostenes
como tambi\'en en los \'ultimos cincuenta a\~nos Miller, Rabin, Pollard, Gries
y otros cient\'ificos han podido desarrollar algoritmos eficientes que permiten
implementar y conseguir resultados combinando la Teor\'ia de N\'umeros y
las Ciencias de la Computaci\'on, lo cual tiene un alto impacto en
el mundo contempor\'aneo tanto en la teor\'ia como en la pr\'actica,
lo cual se ve reflejado en ramas como la Criptograf\'ia y
la Matem\'atica Computacional.
 
\section{Nuestros resultados}
El prop\'osito de este trabajo es doble: 
en primer lugar probaremos la f\'ormula de Selberg en todo rigor
(ver enunciado a continuaci\'on); 
luego diseñaremos y analizaremos la eficiencia de nuestros algoritmos e
implementaremos programas para su verificaci\'on n\'umerica.

\bigskip

Empezamos enunciado la f\'ormula asint\'otica de Selberg. 
En todo lo que sigue los s\'imbolos \(p, q\) se referir\'an a
n\'umeros primos positivos.

\bigskip

\noindent
\textbf{Teorema (La identidad de Selberg)}
\textit{Para todo n\'umero real \(x\) mayor o igual a \(1\)
se cumple la f\'ormula de Selberg}
\[
  \sum_{p \leq x} \ln^2(p) + \sum_{pq \leq x} \ln(p) \ln(q) = 2x\ln(x) + O(x).
\]

\section{Nuestras t\'ecnicas}
Para obtener nuestros resultados,
hemos utilizado herramientas b\'asicas del an\'alisis real,
ejemplos de los cuales tenemos series, sucesiones, continuidad,
l\'imites, derivadas e integrales.
Asimismo, haremos uso de funciones aritm\'eticas
y estimados de estas,
las cuales ser\'an controladas haciendo uso de la notaci\'on \(O\),
la cual tambi\'en utilizaremos para realizar el an\'alisis de complejidad
asint\'otico de los algoritmos.
El algoritmo principal depender\'a de otros algoritmos que realizar\'an
b\'usquedas binarias y una criba lineal para deteminar
los n\'umeros primos en un rango de modo eficiente.
Finalmente, los estimados computacionales ser\'an obtenidos tras realizar una
implementaci\'on de los algoritmos propuestos en el lenguaje GNU C++ 17.

\section{Notaci\'on}
Emplearemos \(f(x) = O(g(x))\) en vez de \(f \in O(g)\)
a pesar de que no se trate de una igualdad de conjuntos sino
pertenencia de una funci\'on a una clase de funciones; 
de la misma manera
trataremos la aritm\'etica entre familias de funciones
con notaci\'on \textit{big \(O\)}.
Los s\'imbolos \(p\) y \(q\), en caso de no especificarse,
har\'an referencia a n\'umeros primos positivos.

\section{Organizaci\'on}
En la secci\'on 2 presentaremos los teoremas y definiciones que no probaremos
pero  son lugar com\'un en \'area,
utilizaremos como referencia \cite{Apostol1976} y \cite{CLRS2009}.
En la secci\'on 3 realizaremos la demostraci\'on de la identidad de Selberg
tras la prueba de ciertos lemas intermedios.
En la secci\'on 4 propondremos los algoritmos CribaLineal,
Buscar\'UltimaPosici\'on y CalcularSuma para poder cumplir con
el objetivo de realizar estimados con el algoritmo EstimarConstante,
el mismo que emplea los tres algoritmos anteriores. 
Todos los algoritmos tendr\'an su respectivo an\'alisis de complejidad asint\'otico.
En la secci\'on 5 implementaremos los algoritmos de la secci\'on anterior
en el lenguaje de programaci\'on C++ y presentaremos tablas con los estimados.
