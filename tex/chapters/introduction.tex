\pretolerance=20000
\tolerance=30000
\setlength{\headheight}{14.61858pt}
\selectlanguage{spanish}
\pagestyle{fancy}

\chapter[Introducci\'on]
{Introducci\'on}
\label{ch:introduction}

\vspace*{1cm}

El Teorema del N\'umero Primo ha sido extensivamente estudiado
\cite{
  Breusch1960,
  Choudhary2017,
  Diamond1982,
  Erdos1949,
  Levinson1969,
  Liu2022,
  Newman1980,
  Pan2023,
  Richter2021,
  Selberg1949,
  Shapiro1959}
(y muchos m\'as).
Este teorema afirma que \(\Psi(x)\) se aproxima a \(x\)
conforme \(x\) se hace arbitrariamente grande.

En \(1949\), Selberg \cite{Selberg1949}
prob\'o este teorema sin uso de An\'alisis Complejo.
Un par de meses luego, Erdős \cite{Erdos1949}
prob\'o el teorema mediante el abuso de estimados tauberianos.
En \(1959\), Shapiro \cite{Shapiro1959}
prueba un par de teoremas tauberianos y equivalencias relacionadas
al trabajo de \cite{Erdos1949},
lo cual desemboc\'o tambi\'en en la prueba del TNP.
En \(1969\), Levinson \cite{Levinson1969} se inspira de \cite{Selberg1949, Breusch1960}
para crear una demostraci\'on elemental contundente haciendo uso de la funci\'on resto.
En \(2017\), Choudhary \cite{Choudhary2017} prueba elementalmente el TNP
reemplazando resultados de \cite{Levinson1969}
por corolarios de \cite{Shapiro1959}.

\section{Nuestros Resultados}
El prop\'osito de este trabajo es doble:
primero, presentamos una nueva prueba elemental del TNP
tras demostrar la f\'ormula de Selberg en todo rigor
(ver enunciado a continuaci\'on);
luego diseñaremos y analizaremos la eficiencia de algoritmos e
implementaremos programas para su verificaci\'on num\'erica.

\bigskip

Empezamos enunciando la f\'ormula asint\'otica de Selberg. 
En todo lo que sigue los s\'imbolos \(p, q\) se referir\'an a
n\'umeros primos positivos.

\bigskip

\noindent
\textbf{Teorema (La identidad de Selberg)}
\textit{Para todo n\'umero real \(x\) mayor o igual a \(1\)
se cumple la f\'ormula de Selberg}
\[
  \sum_{p \leq x} \ln^2(p) + \sum_{pq \leq x} \ln(p) \ln(q) = 2x\ln(x) + O(x).
\]

\section{Nuestras t\'ecnicas}
Para obtener nuestros resultados,
describiremos a continuaci\'on
c\'omo hemos hecho uso de
An\'alisis Real,
Teor\'ia Anal\'itica de N\'umeros,
Algoritmos y
Programaci\'on.

\textbf{Identidad de Selberg.}
\cite{Selberg1949} nos provee esta identidad como
la herramienta m\'as poderosa para alcanzar el premio gordo.
Para demostrarla, usaremos como hoja de ruta los trabajos
\cite{Diamond1982, Choudhary2017}
sin utilizar todas las f\'ormulas asint\'oticas de \cite{Chebyshev1852}
ni haciendo uso de los teoremas tauberianos de \cite{Shapiro1959}.
En su reemplazo, utilizaremos nuestra creatividad y
ciertos resultados de \cite{Mertens1874, TI1951}.

\textbf{Funci\'on Resto}

\textbf{Teorema del N\'umero Primo}


\textbf{Algoritmos}

\section{Notaci\'on}
Emplearemos \(f(x) = O(g(x))\) en vez de \(f \in O(g)\)
a pesar de que no se trate de una igualdad de conjuntos sino
pertenencia de una funci\'on a una clase de funciones; 
de la misma manera
trataremos la aritm\'etica entre familias de funciones
con notaci\'on big-\(O\).
Los s\'imbolos \(p\) y \(q\), en caso de no especificarse,
har\'an referencia a n\'umeros primos positivos.
Asimismo, \(\lg n\) har\'a referencia al logaritmo binario;
es decir, \(\log_2 n\).
% TODO: Anadir polylog(n)

\section{Organizaci\'on}
En la secci\'on 2 presentaremos los teoremas y definiciones que no probaremos
pero  son lugar com\'un en \'area,
utilizaremos como referencia
\cite{Apostol1976},
\cite{CLRS2009} y
\cite{FGST2020}.
En la secci\'on 3 realizaremos la demostraci\'on de la identidad de Selberg
tras la prueba de ciertos lemas intermedios.
En la secci\'on 4 propondremos los algoritmos CribaLineal,
Buscar\'UltimaPosici\'on y CalcularSuma para poder cumplir con
el objetivo de realizar estimados con el algoritmo EstimarConstante,
el mismo que emplea los tres algoritmos anteriores. 
Todos los algoritmos tendr\'an su respectivo an\'alisis de complejidad asint\'otico.
En la secci\'on 5 implementaremos los algoritmos de la secci\'on anterior
en el lenguaje de programaci\'on C++ y presentaremos tablas con los estimados.
