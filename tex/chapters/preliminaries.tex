\pretolerance=20000
\tolerance=30000
\setlength{\headheight}{14.61858pt}
\selectlanguage{spanish}
\pagestyle{fancy}

\chapter[Preliminares Matem\'aticos]
{Preliminares Matem\'aticos}
\label{Chapter02}

En ruta a la identidad de Selberg,
tendremos que recordar algunas definiciones y estimados bastante conocidos.

Dada una funci\'on \(g\) denotamos \(O(g(x))\) al conjunto de funciones
\[
  \begin{gathered}
    O(g(x)) = \{f : \text{existe una constante positiva } c \text{ y un momento } x_0 \text{ tal que} \\
    0 \leq f(x) \leq cg(x) \text{ para todo } x \geq x_0\}.
  \end{gathered}
\]

\begin{theorem}[F\'ormula de sumaci\'on de Euler]
\label{thm:02:01}
Si \(f\) tiene una derivada continua \(f'\) en el intervalo \([a, b]\)
con \(0 < a < b\), entonces se satisface
\[
  \sum_{a < n \leq x} f(n) = \int_a^b f(t) \, dt +
  \int_a^b (t - \floor{t}) f'(t) \, dt +
  f(a)(\floor{a} - a) - f(b)(\floor{b} - b).
\]
\end{theorem}

Sean \(f\) y \(g\) dos funciones aritm\'eticas, 
definimos su \textbf{producto de Dirichlet} como la funci\'on aritm\'etica
\(h\) definida puntualmente por 
\[
  h(n)
  = f * g(n)
  = \sum_{d \mid n} f(d) g \parentheses{\frac{n}{d}}.
\]

La funci\'on \(\boldsymbol{\mu}\) \textbf{de M\"{o}bius} es definida como sigue. 
Primero definimos 
\[
  \mu(1) = 1.
\]
Si \(n > 1\),
expresamos \(n = p_1^{\alpha_1} \cdots p_k^{\alpha_k}\) y definimos
\[
  \mu(n) = 
  \begin{cases}
    (-1)^k &\text{si } \alpha_1 = \cdots = \alpha_k = 1, \\
    0 &\text{en otro caso}. 
  \end{cases}
\]

Para \(n\) entero positivo definimos la funci\'on
\(\boldsymbol{\Lambda}\) \textbf{de Mangoldt} v\'ia 
\[
  \Lambda(n) =
  \begin{cases}
    \ln p &\text{si } n = p^m \text{ para alg\'un } m \geq 1,\\
    0 &\text{en otro caso.}
  \end{cases}
\]

Para \(x > 0\) definimos la funci\'on
\(\boldsymbol{\Psi}\) \textbf{de Chebyshev} con la f\'ormula
\[
  \Psi(x)
  = \sum_{n \leq x} \Lambda(n) 
  = \sum_{\substack{m = 1\\p^m \leq x}}^\infty \sum_p \Lambda(p^m)
  = \sum_{m = 1}^\infty \sum_{p \leq x^{\frac{1}{m}}} \ln p.
\]

Para todo \(x > 0\) definimos la funci\'on \(\boldsymbol{\vartheta}\)
\textbf{de Chebyshev} mediante la ecuaci\'on
\[
  \vartheta(x) = \sum_{p \leq x} \ln p.
\]
