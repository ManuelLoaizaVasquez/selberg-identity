\pretolerance=20000
\tolerance=30000
\setlength{\headheight}{14.61858pt}
\selectlanguage{spanish}
\pagestyle{fancy}

\chapter[La funci\'on resto]
{La funci\'on resto}
\label{ch:remainder}

En este cap\'itulo extraeremos propiedades m\'agicas
de la funci\'on resto definida como \(R(x) = x - \vartheta(x)\)
\cite[secci\'on 3]{Selberg1949}
para posteriormente probar c\'omodamente el PNT
como en la presentaci\'on original de Selberg.

\begin{lemma}
    \label{lem:rlnx1}
    Para todo \(x \geq 1\) tenemos
    \[
        R(x) \ln x = -\sum_{p \leq x} \ln p \, R \parentheses{\frac{x}{p}} + O(x).
    \]
\end{lemma}

\begin{proof}
    Sustituiremos \(\vartheta\) en funci\'on de \(R\) en \cref{lem:theta-2xlnx-ox}
    \begin{align*}
        x\ln x + O(x)
        &= R(x) \ln x
        + \sum_{p \leq x} \ln p \, \vartheta \parentheses{\frac{x}{p}} \\
        &= R(x) \ln
        + \sum_{p \leq x} \ln p \parentheses{\frac{x}{p}
        + R\parentheses{\frac{x}{p}}} \\
        &= R(x) \ln x
        + x \sum_{p \leq x} \frac{\ln p}{p}
        + \sum_{p \leq x} \ln p \, R \parentheses{\frac{x}{p}}.
    \end{align*}
    Podr\'ia parecer que estamos listos, mas aparece un detalle min\'usculo por ajustar.
    Para remediar este defecto trabajamos en su reemplazo con \cite[teorema 2]{Mertens1874}
    \[
        \sum_{p \leq x} \frac{\ln p}{p} = \ln x + O(1).
    \]
    Gracias a ello inferimos
    \[
        R(x) \ln x + x \ln x + O(x)
        + \sum_{p \leq x} \ln p \, R \parentheses{\frac{x}{p}}
        = x \ln x + O(x)
    \]
    y el resultado se desprende de manera inmediata.
\end{proof}

\begin{lemma}
    \label{lem:1-2ln2x-o-lnx}
    Para todo \(x \geq 1\) tenemos
    \[
        \sum_{pq \leq x} \frac{\ln p \ln q}{pq}
        = \frac{1}{2} \ln^2 x + O\parentheses{\ln x}.    
    \]
\end{lemma}

\begin{proof}
    TODO.
\end{proof}

\begin{lemma}
    \label{lem:lnp-lnq-pq-lnpq}
    Para todo \(x \geq 1\) tenemos
    \[
        \sum_{pq \leq x} \frac{\ln p \ln q}{pq \ln pq}
        = \ln x + O(\ln \ln x).   
    \]
\end{lemma}

\begin{proof}
    Definamos \(f: [2, \infty) \to \BR\) con \(f(x) = 1/\ln x\) y
    \(a: \BN \to \BR\) con \(a(n) = \sum_{pq = n} (\ln p \ln q) / pq\).
    Luego aplicamos el \cref{thm:abel}
    y reemplazamos el estimado del \cref{lem:1-2ln2x-o-lnx} para obtener
    \begin{align*}
        \sum_{2 < pq \leq x} \frac{\ln p \ln q}{pq} \cdot \frac{1}{\ln pq}
        &= \sum_{pq \leq x} \frac{\ln p \ln q}{pq} \cdot \frac{1}{\ln x}
        - \int_2^x \parentheses{\sum_{pq \leq t} \frac{\ln p \ln q}{pq}}
        \parentheses{-\frac{1}{t \ln^2 t}} \, dt \\
        &= \parentheses{\frac{1}{2} \ln^2x + O(\ln x)} \frac{1}{\ln x}
        + \int_2^x \parentheses{\frac{\ln^2 t}{2} + O(\ln t)}\parentheses{\frac{1}{t \ln^2 t}} \, dt \\
        &= \frac{1}{2}\ln x + O(1) + \int_{2}^x \frac{1}{2t} \, dt + O\parentheses{\int_2^x \frac{1}{t \ln t} \, dt} \\
        &= \frac{1}{2}\ln x + O(1) + \frac{1}{2}\parentheses{\ln x - \ln 2} + O\parentheses{\ln \ln x - \ln \ln 2} \\
        &= \ln x + O(\ln \ln x)
    \end{align*}
    validando la afirmaci\'on.
\end{proof}

\begin{lemma}
    \label{lem:rlnx2}
    Para todo \(x \geq 1\) tenemos
    \[
        R(x)\ln x
        = \sum_{pq \leq x} \frac{\ln p \ln q}{\ln pq}
        R\parentheses{\frac{x}{pq}} + O\parentheses{x \ln\ln x}.   
    \]
\end{lemma}

\begin{proof}
    TODO.
\end{proof}

\begin{lemma}
    Para todo \(x \geq 1\) tenemos
    \[
        \abs{R(x)}
        \leq \frac{1}{\ln x} \sum_{n \leq x} \abs{R\parentheses{\frac{x}{n}}}
        + O\parentheses{\frac{x\ln \ln x}{\ln x}}.
    \]
\end{lemma}

\begin{proof}
    Sumamos las expresiones obtenidas en \cref{lem:rlnx1} y \cref{lem:rlnx2}
\end{proof}

\begin{lemma}
    Para todo \(x \geq 1\) tenemos
    \[
        \sum_{n \leq x} \frac{R(n)}{n^2} = O(1).    
    \]
\end{lemma}

\begin{proof}
    TODO.
    Sumas parciales de \cite[teorema 2]{Mertens1874} en
    \[
        \sum_{n \leq x} \frac{\vartheta(n)}{n^2} = \ln x + O(1).
    \]
\end{proof}

\begin{lemma}
    Existe \(K_1 > 0\) tal que para \(x' > x > 4\) tenemos
    \[
        \abs{\sum_{x \leq n \leq x'} \frac{R(n)}{n^2}} < K_1.
    \]
\end{lemma}

\begin{proof}
    TODO.
\end{proof}

\begin{lemma}
    Para \(x' > x > 4\) existen \(y \in \brackets{x, x'}\)
    y \(K_2 \geq 1\) tales que
    \[
        \abs{\frac{R(y)}{y}} < \frac{K_2}{\ln(x'/x)}.
    \]
\end{lemma}

\begin{proof}
    TODO.
\end{proof}

\begin{lemma}
    Para \(\delta < 1\) y \(x > 4\) existe
    \(y \in \brackets{x, e^{K_2 / \delta} x}\) tal que
    \(\abs{R(y)} < \delta y\).
\end{lemma}

\begin{proof}
    TODO.
\end{proof}

\begin{lemma}
    Para \(y' > y > 4\) tal que \(y/2 \leq y' \leq 2y\) tenemos
    \[
        \abs{R(y')} \leq \abs{R(y)} + \abs{y' - y} + O\parentheses{\frac{y'}{\ln y'}}.    
    \]
\end{lemma}

\begin{proof}
    TODO.
\end{proof}

\begin{lemma}
    Existe \(y \in \parentheses{x, e^{K_2/\delta}x}\) tal que
    para todo \(y' \in \brackets{y/2, 2y}\) con
    \(\delta < 1\) y \(x > 4\) tenemos
    \[
        \abs{\frac{R(y')}{y'}} < 2\delta + \abs{1 + \frac{y'}{y}} + \frac{K_3}{\ln x}.
    \]
\end{lemma}

\begin{proof}
    TODO.
\end{proof}

\begin{lemma}
    Para \(\delta < 1\), \(x > 4\) y \(x > e^{K_3/\delta}\),
    el intervalo \(\parentheses{x, e^{K_2/\delta}x}\) siempre contendr\'a
    un subintervalo \(\parentheses{y, e^{\delta/2}y}\) tal que
    \(\abs{R(z)} < 4 \delta z\) para todo \(z\) en dicho subintervalo.
\end{lemma}

\begin{proof}
    TODO.
\end{proof}
