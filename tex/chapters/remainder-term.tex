\pretolerance=20000
\tolerance=30000
\setlength{\headheight}{14.61858pt}
\selectlanguage{spanish}
\pagestyle{fancy}

\chapter[La funci\'on resto]
{La funci\'on resto}
\label{ch:remainder}

En este cap\'itulo extraeremos propiedades m\'agicas
de la funci\'on resto definida como \(R(x) = x - \vartheta(x)\)
\cite[secci\'on 3]{Selberg1949}
para posteriormente probar c\'omodamente el PNT
como en la presentaci\'on original de Selberg.

\begin{lemma}
    \label{lem:rlnx1}
    Para todo \(x \geq 1\) tenemos
    \[
        R(x) \ln x = -\sum_{p \leq x} \ln p \, R \parentheses{\frac{x}{p}} + O(x).
    \]
\end{lemma}

\begin{proof}
    Sustituiremos \(\vartheta\) en funci\'on de \(R\) en \cref{lem:theta-2xlnx-ox}
    \begin{align*}
        x\ln x + O(x)
        &= R(x) \ln x
        + \sum_{p \leq x} \ln p \, \vartheta \parentheses{\frac{x}{p}} \\
        &= R(x) \ln
        + \sum_{p \leq x} \ln p \parentheses{\frac{x}{p}
        + R\parentheses{\frac{x}{p}}} \\
        &= R(x) \ln x
        + x \sum_{p \leq x} \frac{\ln p}{p}
        + \sum_{p \leq x} \ln p \, R \parentheses{\frac{x}{p}}.
    \end{align*}
    Podr\'ia parecer que estamos listos, mas aparece un detalle min\'usculo por ajustar.
    Para remediar este defecto trabajamos en su reemplazo con \cite[teorema 2]{Mertens1874}
    \[
        \sum_{p \leq x} \frac{\ln p}{p} = \ln x + O(1).
    \]
    Gracias a ello inferimos
    \[
        R(x) \ln x + x \ln x + O(x)
        + \sum_{p \leq x} \ln p \, R \parentheses{\frac{x}{p}}
        = x \ln x + O(x)
    \]
    y el resultado se desprende de manera inmediata.
\end{proof}
