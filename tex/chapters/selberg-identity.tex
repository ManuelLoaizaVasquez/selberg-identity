\pretolerance=20000
\tolerance=30000
\setlength{\headheight}{14.61858pt}
\selectlanguage{spanish}
\pagestyle{fancy}

\chapter[La Identidad de Selberg]
{La Identidad de Selberg}
\label{ch:selberg}

\begin{lemma}
  \label{lem:harmonic}
  Para todo \(x \geq 1\) tenemos
  \[
    \sum_{n \leq x} \frac{1}{n} = \ln x + \gamma + O\left(\frac{1}{x}\right),
  \]
  aqu\'i \(\gamma \approx 0.52\) es una constante conocida como
  \textbf{la constante de Euler}.
\end{lemma}

\begin{proof}
  La funci\'on \(f : [1, \infty) \to \BR\) con \(f(x)=1/x\) es continua y
  diferenciable en toda la recta conque podemos aplicar \Cref{thm:euler}
  en cualquier intervalo \([2, k]\)
  y as\'i obtener
  \[
    \sum_{n = 2}^k \frac{1}{n} =
    \int_1^k \frac{dt}{t} + \int_{1}^k (t - \floor{t})\parentheses{\frac{1}{-t^2}} \, dt
    = \ln k - \int_1^k \frac{t - \floor{t}}{t^2} \, dt,
  \]
  lo cual conduce de inmediato a 
  \[
    \sum_{n = 1}^k \frac{1}{n} - \ln k
    = 1 - \int_1^k \frac{t - \floor{t}}{t^2} \, dt.
  \]
  
  Para analizar qu\'e ocurre cuando \(k \to \infty\) escribimos
  \[
    \gamma = \lim_{k \to \infty} \parentheses{\sum_{n = 1}^k \frac{1}{n} - \ln k}
    = 1 - \lim_{k \to \infty}  \parentheses{\int_1^k \frac{t - \floor{t}}{t^2}} \, dt
    = 1 - \int_1^\infty \frac{t - \floor{t}}{t^2} \, dt,
    \eqlabel{eq:floor-int}
  \]
  l\'imite que existe, pues al tenerse 
  \begin{align*}
    \int_1^k \frac{t - \floor{t}}{t^2} \, dt
    \le \int_1^k \frac{1}{t^2} \, dt
    = 1 -\frac{1}{k}
    \le 1
  \end{align*}
  la convergencia queda garantizada por monotonicidad.
  
  Finalmente, para establecer la f\'ormula anunciada
  reemplazamos \eqref{eq:floor-int} tras el uso de \Cref{thm:euler}
  \begin{align*}
    \sum_{n \leq x} \frac{1}{n} &= \int_1^x \frac{1}{t} \, dt +
    \int_1^x \parentheses{t - \floor{t}} \parentheses{-\frac{1}{t^2}} \, dt
    - \parentheses{x - \floor{x}} \parentheses{\frac{1}{x}} + 1 \\
    &= \ln x - \int_1^x \frac{t - \floor{t}}{t^2} \, dt + 1 - \frac{x - \floor{x}}{x} \\
    &= \ln x - \int_1^x \frac{t - \floor{t}}{t^2} \, dt
    + 1 - \int_1^{\infty} \frac{t - \floor{t}}{t^2} \, dt +
    \int_1^{\infty} \frac{t - \floor{t}}{t^2} \, dt
    - \frac{x - \floor{x}}{x} \\
    &= \ln x + \gamma + \int_x^{\infty} \frac{t - \floor{t}}{t^2} \, dt
    - \frac{x - \floor{x}}{x} \\
    &\leq \ln x + \gamma + \int_x^{\infty} \frac{1}{t^2} \, dt - \frac{x - \floor{x}}{x} \\
    &= \ln x + \gamma + \frac{1 - \parentheses{x - \floor{x}}}{x}
  \end{align*}
  alcanzando nuestro objetivo.
\end{proof}

\begin{lemma}
  \label{lem:sum-ln}
  Para todo \(x \geq 1\) tenemos
  \[
    \sum_{n \leq x} \ln n = x\ln x - x + O\parentheses{\ln x}.
  \]
\end{lemma}

\begin{proof}
  Esta vez utilizamos \Cref{thm:euler} en
  \(f : [1, \infty) \to \BR\) con \(f(x) = \ln x\),
  continua y diferenciable en toda la recta real positiva:  
  \begin{align*}
    \sum_{n \leq x} \ln x &= \int_1^x \ln t \, dt +
    \int_1^x  \frac{t - \floor{t}}{t} \, dt + \parentheses{\floor{x} - x} \ln x  \\
    &= x \ln x - x + 1 + \int_1^x \frac{t - \floor{t}}{t} \, dt
    + \parentheses{\floor{x} - x} \ln x \\
    &\leq x \ln x - x + 1 + \int_1^x \frac{1}{t} \,dt + \ln x \\
    &= x \ln x - x + 1 + 2 \ln x.
  \end{align*}
  Como \(1\) est\'a dominado por \(\ln x\),
  se cumple \(\sum_{n \leq x} \ln n = x\ln x - x + O(\ln x)\), 
  lo anunciado.
\end{proof}

\begin{lemma}
  \label{lem:sum-sum-f}
  Para toda funci\'on aritm\'etica \(f\) se cumple
  \[
    \sum_{n \leq x} \sum_{d \mid n} f(d) =
    \sum_{n \leq x} f(n) \floor{\frac{x}{n}}.
  \]
\end{lemma}

\begin{proof}
  Sean \(f\) y \(g\) dos funciones aritm\'eticas,
  \(F\) y \(G\) sus respectivas cumulativas;
  es decir, \(F(x) = \sum_{n \leq x} f(x)\) y
  \(G(x) = \sum_{n \leq x} g(x)\).
  La cumulativa del producto de Dirichlet de \(f\) y \(g\) est\'a dada por 
  \[
    \sum_{n \leq x} f * g (n) = \sum_{n \leq x} \sum_{cd = n} f(c)g(d)
    = \sum_{c \leq x} \sum_{d \leq x/c} f(c) g(d)
    = \sum_{c \leq x} f(c) \sum_{d \leq x/c} g(d)
    = \sum_{c \leq x} f(c) G\parentheses{\frac{x}{c}}.
    \eqlabel{eq:sum-f-G}
  \]

  En particular, cuando \(g = \mathbf{1}\), su cumulativa es
  \[
    \sum_{n \leq x} \mathbf{1}(n) = \sum_{n \leq x} 1 = \floor{x}.
  \]
  De este modo, al introducir \(G(x) = \floor{x}\) en \eqref{eq:sum-f-G} se logra
  \[
    \sum_{n \leq x} \sum_{d \mid n} f(d)
    = \sum_{n \leq x} f * \mathbf{1} (n)
    = \sum_{n \leq x} f(n) \floor{\frac{x}{n}},
  \]
  lo buscado.
\end{proof}

\begin{lemma}
  \label{lem:floor-is-x-o1}
  Para todo n\'umero real \(x\) tenemos
  \[
    \floor{x} = x + O(1).
  \]
\end{lemma}

\begin{proof}
  Sea \(x = n + r\) un n\'umero real no negativo, con
  \(n\) entero y \(0 \leq r < 1\).
  De esta manera, por definici\'on de m\'aximo entero obtenemos
  \(
    \floor{x} = n = x - r = x + O(1)
  \)
  concluyendo trivialmente con la propiedad.
\end{proof}

\begin{lemma}
  [\cite{Chebyshev1852}]
  \label{lem:psi-is-ox}
  Para todo \(x \geq 1\) tenemos
  \[
    \Psi(x) = O(x).
  \]
\end{lemma}

\begin{proof}
  Usaremos el desarrollo de los estimados de Chebyshev de \cite{Diamond1982}
  \[
    A \leq \liminf \frac{\Psi(x)}{x} \leq \limsup \frac{\Psi(x)}{x} \leq \frac{6A}{5}
  \]
  con
  \[
    A = -\frac{\ln 1}{1} + \frac{\ln 2}{2} + \frac{\ln 3}{3} + \frac{\ln 5}{5} - \frac{\ln 30}{30}
    \approx 0.9212920229340907809134...
  \]
  Reescribimos la parte derecha de la desigualdad con valor absoluto ya que es una funci\'on positiva cual
  \(
    \limsup \abs{\Psi(x)}/x \leq 6A/5.
  \)
  Por definici\'on, existe \(n_0\) a partir del cual se tiene
  \(
    \abs{\Psi(x)}/x \leq 6A/5,
  \)
  para todo \(x \geq n_0\).
  Como ello equivale a
  \(
    \abs{\Psi(x)} \leq (6A/5) x,
  \)
  se consigue \(\Psi(x) = O(x)\).
\end{proof}

\begin{lemma}
  \label{lem:mangoldt-is-mu-ln}
  La funci\'on de Mangoldt se puede expresar como el producto de Dirichlet
  \(
    \Lambda = \mu * \ln.
  \)
\end{lemma}

\begin{proof}
  Esta f\'ormula equivale a
  \(\Lambda * 1 = \ln\)
  v\'ia \Cref{thm:mobius}.
\end{proof}

\begin{lemma}
  [\cite{Mertens1874}]
  \label{lem:mangoldt-n-lnx-o1}
  Para todo $x \geq 1$ tenemos
  \[
    \sum_{n \leq x} \frac{\Lambda(n)}{n} = \ln x + O(1).
  \]
\end{lemma}

\begin{proof}
  El desarrollo de
  \(
    \ln = \Lambda * 1
  \)
  cual 
  \(
    \ln n = \sum_{d \mid n} \Lambda(d)
  \)
  lleva a
  \[
    \sum_{n \leq x} \ln n = \sum_{n \leq x} \sum_{d \mid n} \Lambda(d).
  \]
  Una aplicaci\'on directa de \Cref{lem:sum-sum-f} deriva en
  \[
    \sum_{n \leq x} \ln n = \sum_{n \leq x} \Lambda(n) \floor{\frac{x}{n}}.
  \]
  De ac\'a, en uso de \Cref{lem:floor-is-x-o1} conseguimos
  \begin{align*}
    \sum_{n \leq x} \ln n &= \sum_{n \leq x} \Lambda(n) \parentheses{\frac{x}{n} + O(1)}
    = x \sum_{n \leq x} \frac{\Lambda(n)}{n} + O\parentheses{\sum_{n \leq x} \Lambda(x)} \\
    &= x \sum_{n \leq x} \frac{\Lambda(n)}{n} + O(\Psi(x))
    = x \sum_{n \leq x} \frac{\Lambda(n)}{n} + O(x)
  \end{align*}
  dado que, por \Cref{lem:psi-is-ox}, \(\Psi(x) = O(x)\) claramente implica
  \(
    O(\Psi(x))=O(x)
  \).
  Si aplicamos \Cref{lem:sum-ln} al lado izquierdo desembocamos en
  \[
    x\ln x - x + O(\ln x) = x \sum_{n \leq x} \frac{\Lambda(n)}{n} + O(x). 
  \]
  Al despejar obtenemos
  \[
    \sum_{n \leq x} \frac{\Lambda(n)}{n}
    = \ln x - 1 + O(1) + O\parentheses{\frac{\ln x}{x}}
    = \ln x + O(1),
  \]
  pues
  \(
    -1 + O(1)+ O(\ln x/x)
  \)
  es acotado.
\end{proof}

\begin{lemma}
  \label{lem:sum-mu-g}
  Para \(f, g : [1, \infty) \to \BR\)
  sujetos a  \(g(x) = \sum_{n \leq x} f\parentheses{\frac{x}{n}} \ln x\) tenemos
  \[
    \sum_{n \leq x} \mu(n) g\parentheses{\frac{x}{n}}
    = f(x) \ln(x) + \sum_{n \leq x} f\parentheses{\frac{x}{n}} \Lambda(n).
  \]
\end{lemma}

\begin{proof}
  Desarrollemos la sumatoria que queremos analizar
  \begin{align*}
    \sum_{n \leq x} \mu(n) g\parentheses{\frac{x}{n}}
    &= \sum_{n \leq x} \mu(n)
    \sum_{m \leq x/n} f\parentheses{\frac{x}{nm}} \ln \parentheses{\frac{x}{n}} \\
    &= \sum_{nm \leq x} \mu(n) \ln\parentheses{\frac{x}{n}} f \parentheses{\frac{x}{nm}} \\
    &= \sum_{c \leq x} f\parentheses{\frac{x}{c}}
    \sum_{d \mid c} \mu(d) \ln\parentheses{\frac{x}{d}} \\
    &= \sum_{n \leq x} f\parentheses{\frac{x}{n}}
    \sum_{d \mid n} \mu(d)
    \brackets{\ln \parentheses{\frac{x}{n}} + \ln\parentheses{\frac{n}{d}}} \\
    &= \brackets{\sum_{n \leq x} f\parentheses{\frac{x}{n}} \ln\parentheses{\frac{x}{n}} \sum_{d \mid n} \mu(d)}
    +  \brackets{\sum_{n \leq x} f\parentheses{\frac{x}{n}} \sum_{d \mid n} \mu(d) \ln\parentheses{\frac{n}{d}}}.\\
    &= f(x) + \ln{x} + \sum_{n \leq x} f\parentheses{\frac{x}{n}} (\mu * \ln)(n).
  \end{align*}
  Con ello, finalmente,
  utilizamos \Cref{lem:mangoldt-is-mu-ln} para concluir lo deseado.
\end{proof}

\begin{lemma}
  \label{lem:ln2x-is-sqrtx}
  Para todo \(x \geq 1\) tenemos
  \[
    \ln^2 x = O(\sqrt{x}).
  \]
\end{lemma}

\begin{proof}
  Como sabemos que para todo \(x \geq 1\) se cumple que \(x > \ln x\)
  (v\'ia an\'alisis de la  derivada de \(x - \ln x\)), se obtiene
  \[
    \ln^2 x
    = \ln^2 \parentheses{\parentheses{x^{1/4}}^4}
    = 16 \ln^2 \parentheses{x^{1/4}}
    < 16 (x^{1/4})^2
    = 16 \sqrt{x}.
  \]
\end{proof}

\begin{lemma}
  \label{lem:psi-2xlnx-ox}
  Para todo \(x \geq 1\) tenemos
  \[
    \Psi(x) \ln x + \sum_{n \leq x}
    \Psi\parentheses{\frac{x}{n}}\Lambda(n) = 2x\ln x + O(x).
  \]
\end{lemma}

\begin{proof}
  Para utilizar \Cref{lem:sum-mu-g},
  definimos convenientemente \(f : [1, \infty) \to \BR\) con
  \[
    f(x) = \Psi(x) - x + \gamma + 1.
  \]
  Primero le brindaremos a
  \(
    g(x) = \sum_{n \leq x} f\parentheses{x / n}\ln x
  \)
  una expansi\'on diferente cual es

  \begin{align*}
    g(x)
    &= \sum_{n \leq x} f\parentheses{\frac{x}{n}}\ln x
    = \sum_{n \leq x} \parentheses{\Psi\parentheses{\frac{x}{n}} - \frac{x}{n} + \gamma + 1} \ln x \\
    &= \sum_{n \leq x} \Psi\parentheses{\frac{x}{n}}\ln x - x \ln x
    \sum_{n \leq x} \frac{1}{n} + (\gamma + 1) \ln x \sum_{n \leq x} 1.
    \eqlabel{eq:triple-sum}
  \end{align*}
  
  Analicemos por separado cada sumatoria de \eqref{eq:triple-sum}.
  
  La primera resulta ser 
  \begin{align*}
    \sum_{n \leq x} \Psi\parentheses{\frac{x}{n}}
    = \sum_{n \leq x} \sum_{d \leq x/n} \Lambda(d)
    = \sum_{n \leq x} \sum_{d \mid n} \Lambda(d)
    =  \sum_{n \leq x} (\Lambda * 1)(n)
    = \sum_{n \leq x} \ln n,
  \end{align*}
  de tipo \(x\ln x - x + O(\ln x)\) por \Cref{lem:sum-ln}.
  Al multiplicar el logaritmo obtenemos la expresi\'on
  \[
    \sum_{n \leq x} \Psi\parentheses{\frac{x}{n}}\ln x
    = x \ln^2 x - x \ln x + O(\ln^2 x).
  \]
  
  Para la segunda recurrimos a \Cref{lem:harmonic} y logramos
  \[
    -x \ln x \sum_{n \leq x} \frac{1}{n}
    = -x \ln x \parentheses{\ln x + \gamma + O\parentheses{\frac{1}{x}}}
    = -x \ln^2 x - \gamma x \ln x + O(\ln x).
  \]
  
  Para la tercera necesitamos \Cref{lem:floor-is-x-o1}: 
  \[
    (\gamma + 1) \ln x \sum_{n \leq x} 1
    = (\gamma + 1) \ln x \floor{x}
    = (\gamma + 1) \ln x (x + O(1))
    = (\gamma + 1) x \ln x + O(\ln x).
  \]
  
  Finalmente, juntamos los tres resultados y obtenemos
  \begin{align*}
    g(x)
    &= x \ln^2 x - x \ln x + O(\ln^2 x)
    - x \ln^2 x - \gamma x \ln x + O(\ln x)
    + (\gamma + 1) x \ln x + O(\ln x) \\
    &= O(\ln^2 x) + O(\ln x)
    = O(\ln^2 x).
  \end{align*}
  
  De \Cref{lem:sum-mu-g} obtenemos entonces
  \[
    \sum_{n \leq x} \mu(n) g\parentheses{\frac{x}{n}}
    = \parentheses{\Psi(x) - x + \gamma + 1} \ln x +
    \sum_{n \leq x} \parentheses{\Psi\parentheses{\frac{x}{n}} - \frac{x}{n} + \gamma + 1} \Lambda(n).
    \eqlabel{eq:psi-lnx-sum-psi-mangoldt}
  \]
  
  El remate consiste en analizar ambos miembros de la desigualdad por separado.
  Utilizamos la desigualdad triangular el hecho de que se cumple
  \(g(x) = O(\ln^2 x)\) para obtener a la izquierda
  \[
    \abs{\sum_{n \leq x} \mu(n) g\parentheses{\frac{x}{n}}}
    \leq \sum_{n \leq x} \abs{g\parentheses{\frac{x}{n}}}
    = O\parentheses{\sum_{n \leq x} g\parentheses{\frac{x}{n}}}
    = O\parentheses{\sum_{n \leq x} \ln^2 \parentheses{\frac{x}{n}}}.
  \]
  Con esta expresi\'on, \Cref{lem:ln2x-is-sqrtx} permite conseguir
  \[
    \sum_{n \leq x} \mu(n) g\parentheses{\frac{x}{n}}
    = O\parentheses{\sum_{n \leq x} \sqrt{\frac{x}{n}}}
    = O\parentheses{\sqrt{x} \sum_{n \leq x} \frac{1}{\sqrt{n}}}
    = O\parentheses{\sqrt{x}\cdot \sqrt{x}}
    = O(x).
    \eqlabel{eq:left-expression}
  \]
  
  El t\'ermino de la derecha en \eqref{eq:psi-lnx-sum-psi-mangoldt} lo reordenamos cual
  \[
    \Psi(x)\ln x
    + \sum_{n \leq x} \Psi\parentheses{\frac{x}{n}}\Lambda(n)
    - x \ln x - x \sum_{n \leq x} \frac{\Lambda(n)}{n}
    + (\gamma + 1)\Psi(x).\eqlabel{eq:reordered-right-expression}
  \]
  Merced a \Cref{lem:psi-is-ox} y \Cref{lem:mangoldt-n-lnx-o1} reducimos \eqref{eq:reordered-right-expression}
  \begin{align*}
    \Psi(x)\ln x +
    \sum_{n \leq x} \Psi\parentheses{\frac{x}{n}}\Lambda(n)
    - x \ln x - x(\ln x + O(1)) + O(x) \\
    = \Psi(x)\ln x +
    \sum_{n \leq x} \Psi\parentheses{\frac{x}{n}}\Lambda(n) - 2x \ln x + O(x).
    \eqlabel{eq:right-expression}
  \end{align*}
  Finalmente, igualamos \eqref{eq:left-expression} con \eqref{eq:right-expression}
  \[
    \Psi(x)\ln x +
    \sum_{n \leq x} \Psi\parentheses{\frac{x}{n}}\Lambda(n)
    - 2x \ln x + O(x)
    = O(x),
  \]
  equivalente a lo aseverado.
\end{proof}

Tatuzawa e Iseki \cite{TI1951}
trabajaron \Cref{lem:psi-2xlnx-ox} haciendo uso de integrales.
Nosotros hemos evitado aquello astutamente haciendo
uso de nuestra querida desigualdad triangular.

\begin{lemma}
  \label{lem:psi-is-theta-osqrtxlnx}
  Para todo \(x \geq 1\) tenemos
  \[
    \Psi(x) = \vartheta(x) + O\parentheses{\sqrt{x}\ln x}.
  \]
\end{lemma}

\begin{proof}
  Directo de la definici\'on observamos que se cumple 
  \[
    \Psi(x) = \sum_{n = 1}^\infty \vartheta\parentheses{x^{1/n}}.
  \]
  Notemos que al mismo tiempo esta sumatoria tiene apenas
  una cantidad finita de t\'erminos efectivos puesto que la funci\'on \(\vartheta\)
  solo tiene sentido cuando es evaluada en valores mayores o iguales a \(2\).
  Para un \(x\) espec\'ifico, hallamos ese momento \(m\) mediante la desigualdad
  \(x^{1/m} \geq 2\),
  pues elev\'andola al cuadrado
  \(x^{2/m} \geq 4 > e\)
  y aplic\'andole logaritmo
  \((2/m)\ln x > 1\)
  obtenemos
  \(2 \ln x > m\).
  Notamos que para valores mayores que
  \(m = \floor{2 \ln x}\) los constituyentes de la suma son nulos.
  Ahora podemos escribir a \(\Psi\) como 
  \[
    \Psi(x) = \vartheta(x) + \vartheta(x^{1/2}) + \cdots + \vartheta(x^{1/m})
    = \vartheta(x) + \sum_{n = 2}^m \vartheta(x^{1/n}).
  \]
  Para el an\'alisis de \(\sum_{n = 2}^m \vartheta(x^{1/n})\) desdoblamos
  \[
    \sum_{n = 2}^m \vartheta(x^{1/n})
    = \sum_{n = 2}^m \sum_{p \leq x^{1/n}} \ln p.
  \]
  Trataremos de darle forma manipulativa sencilla.
  Si un n\'umero primo \(p\) ser\'a inmiscuido en la sumatoria,
  su logaritmo contribuir\'a a la sumatoria tantas veces como las ra\'ices de \(x\) lo permitan:
  entre \(1\) y \(k\), donde \(k\) es el m\'aximo entero que obedece \(p^k \le x\).
  F\'acilmente hallamos que este m\'aximo est\'a dado por \(k = \floor{\ln x / \ln p}\).
  
  Por su parte, para forzar por lo menos \(p^2 \le x\),
  se necesita \(p \le \sqrt{x}\), detalle importante que aprovecharemos.
  
  Ahora analicemos la forma equivalente de la doble sumatoria
  \begin{align*}
    \sum_{n = 2}^m \sum_{p^n \leq x} \ln p
    &= \sum_{p \leq \sqrt{x}} \, \sum_{2 \leq n \leq \floor{\ln x / \ln p}} \ln p
    = \sum_{p \leq \sqrt{x}} \ln p \sum_{2 \leq n \leq \floor{\ln x / \ln p}} 1
    \leq \sum_{p \leq \sqrt{x}} \ln p \floor{\frac{\ln x}{\ln p}} \\
    &\leq \sum_{p \leq \sqrt{x}} \ln p \parentheses{\frac{\ln x}{\ln p}}
    = \sum_{p \leq \sqrt{x}} \ln x
    = \ln x \sum_{p \leq \sqrt{x}} 1
    \leq \ln x \sum_{n \leq \sqrt{x}} 1
    \leq \ln x \sqrt{x}, 
  \end{align*}
  lo que permite concluir \(\sum_{n = 2}^m \vartheta(x^{1/n}) = O(\sqrt{x} \ln x)\).
  Con lo anterior queda establecida la relaci\'on \(\Psi(x) = \vartheta(x) + O(\sqrt{x} \ln x)\).
\end{proof}

\begin{lemma}
  \label{lem:sum-lnp}
  La serie
  \[
    \sum_{p = 2}^{\infty} \frac{\ln p}{p (p - 1)}
  \]
  tomada sobre los primos converge.
\end{lemma}

\begin{proof}
  El primer paso es notar que el l\'imite
  \[
    \lim_{n \to \infty} \frac{\ln n}{n (n - 1)} n^{3/2}
  \]
  vale \(0\) como se deduce al descomponer 
  \[
    \frac{\ln n}{n (n - 1)} n^{3/2}
    = \parentheses{\frac{\ln n}{\sqrt{n}}}\parentheses{\frac{n}{n - 1}}.
  \]
  
  Analicemos el l\'imite de lo que est\'a dentro del par\'entesis de la izquierda
  \[
    \lim_{n \to \infty} \frac{\ln n}{\sqrt{n}} = \frac{\infty}{\infty}.
  \]
  Como este es de la forma \(\infty / \infty\), aplicamos la regla de L'Hospital
  \[
    \lim_{n \to \infty} \frac{\ln n}{\sqrt{n}}
    = \lim_{n \to \infty}\frac{1/n}{1/(2\sqrt{n})}
    = \lim_{n \to \infty} \frac{2}{\sqrt{n}} = 0.
  \]
  
  Ahora analicemos el l\'imite de lo que est\'a dentro del par\'entesis de la derecha
  \[
    \lim_{n \to \infty} \frac{n}{n - 1}
    = \lim_{n \to \infty} \frac{1}{1 - 1/n}
    = \frac{1}{1 - 0} = 1.
  \]
  
  Como ambos l\'imites existen, por aritm\'etica de l\'imites obtenemos
  \[
    \lim_{n \to \infty} \frac{\ln n}{n (n - 1)} n^{3/2}
    = \parentheses{\lim_{n \to \infty} \frac{\ln n}{\sqrt{n}}}
    \parentheses{\lim_{n \to \infty} \frac{n}{n - 1}}
    = 0 \cdot 1
    = 0.
  \]
  
  Por supuesto, lo mismo es v\'alido si crecemos a lo largo de primos, conque se tiene 
  \[
    \lim_{p \to \infty} \frac{\ln p}{p (p - 1)} p^{3/2}=0.
  \]
  
  Por definici\'on entonces, dado \(\eps > 0\), existe un \(n_0\) a partir del cual se tiene
  \[
    \frac{\ln p}{p (p - 1)} p^{3/2} < \eps
  \]
  
  De este modo, al hacer \(p_0 = \floor{n_0} + 1\),
  sabemos que para todo \(p \geq p_0\) se tendr\'a 
  \[
    \sum_{p \geq p_0} \frac{\ln p}{p (p - 1)}
    < \sum_{p \geq p_0} \frac{\eps}{p^{3/2}}.
  \]
  
  Como el menor primo es \(2\), logramos el estimado
  \[
    \sum_{p \geq p_0} \frac{\ln p}{p (p - 1)}
    < \sum_{p \geq p_0} \frac{\eps}{p^{3/2}}
    \le \int_{1}^\infty \frac{\eps}{x^{3/2}} \, dx
    < 2 \eps.
  \]
  
  Esto, por supuesto, lleva a
  \[
    \sum_{p = 2}^\infty \frac{\ln p}{p (p - 1)}
    < \sum_{p < p_0} \frac{\ln p}{p (p - 1)} + 2 \eps,
  \]
  lo que equivale a la convergencia absoluta de la serie.
\end{proof}

\begin{lemma}
  \label{lem:theta-2xlnx-ox}
  Para todo \(x \geq 1\) tenemos
  \[
    \vartheta(x)\ln x
    + \sum_{p \leq x} \vartheta\parentheses{\frac{x}{p}}\ln p
    = 2x\ln x + O(x).
  \]
\end{lemma}

\begin{proof}
  El primer paso es comparar la sumatoria con otra m\'as a tono con nuestros intereses:
  \begin{align*}
    \sum_{n \leq x} \Psi \parentheses{\frac{x}{n}} \Lambda(n)
    - \sum_{p \leq x} \vartheta \parentheses{\frac{x}{p}} \ln p
    &= \sum_{n \leq x} \sum_{m \leq x/n} \Lambda(m) \Lambda(n)
    - \sum_{p \leq x} \sum_{q \leq x / p} \ln q \ln p \\
    &= \sum_{nm \leq x} \Lambda(n) \Lambda(m)
    - \sum_{pq \leq x} \ln p \ln q.
  \end{align*}
  En el primero de los dos sumandos sobrevivientes,
  la funci\'on de Mangoldt solo act\'ua sobre las potencias de los primos.
  En particular, todas las combinaciones de primos con potencias iguales a uno
  van de la mano con la sumatoria que estamos restando a la derecha.
  De esta manera, apenas sobreviven aquellos t\'erminos
  con al menos uno de los exponentes mayor o igual a dos.
  De este modo, se consigue
  \begin{align*}
    \sum_{nm \leq x} \Lambda(n) \Lambda(m)
    - \sum_{pq \leq x} \ln p \ln q
    \leq \sum_{\substack{p^n q^m \leq x \\ n \geq 2 \\ m \geq 1}} \ln p \ln q
    + \sum_{\substack{p^n q^m \leq x \\ n \geq 1 \\ m \geq 2}} \ln p \ln q \\
    = 2 \sum_{\substack{p^n q^m \leq x \\ n \geq 2 \\ m \geq 1}} \ln p \ln q
    = 2 \sum_{\substack{p^n \leq x \\ n \geq 2}} \ln p \sum_{\substack{q^m \leq x / p^n \\ m \geq 1}} \ln q
    = O\parentheses{\sum_{\substack{p^n \leq x \\ n \geq 2}} \ln p \, \Psi\parentheses{\frac{x}{p}}},
  \end{align*}
  puesto que tras desigualdad contamos por partida doble
  aquellos pares con ambos exponentes al menos dos y ello contribuyen con valores positivos.
  % Debido a que los primos deben tomarse por pares,
  % podemos asegurarnos de este exponente en el primer n\'umero primo y que el exponente del segundo n\'umero primo tome cualquier valor
  % \begin{align*}
  % \sum_{n \leq x} \Psi\left(\frac{x}{n}\right)\Lambda(n) - \sum_{p \leq x} \vartheta\left(\frac{x}{p}\right)\ln p
  % &= O\left(\sum_{\substack{p^n q^m \leq x \\ n \geq 2, m \geq 1}} \ln p \ln q\right) \\
  % &= O\left(\sum_{\substack{p^n \leq x \\ n \geq 2}} \ln p \sum_{\substack{q^m \leq \frac{x}{p^n} \\ m \geq 1}} \ln q\right) \\
  % &= O\left(\sum_{\substack{p^n \leq x \\ n \geq 2}} \ln p \, \Psi\left(\frac{x}{p^n}\right)\right).
  % \end{align*}
  Para continuar, utilizamos \Cref{lem:psi-is-ox} en la \'ultima igualdad y logramos 
  \[
    \sum_{n \leq x} \Psi\parentheses{\frac{x}{n}}\Lambda(n)
    - \sum_{p \leq x} \vartheta\parentheses{\frac{x}{p}} \ln p
    = O\parentheses{\sum_{\substack{p^n \leq x \\ n \geq 2}} \ln p \, \frac{x}{p^n}}
    = O\parentheses{x \sum_{\substack{p^n \leq x \\ n \geq 2}} \frac{\ln p}{p^n}}.
  \]
  
  Llegado este punto, nuevamente notamos que la contribuci\'on
  en la cola es exclusiva de los primos sujetos a \(p^2 \le x\).
  Asimismo, los t\'erminos de la sumatoria est\'an dominados por una serie geom\'etrica,
  por lo que conseguimos
  \begin{align*}
    O\parentheses{x \sum_{\substack{p^n \leq x \\ n \geq 2}} \frac{\ln p}{p^n}}
    &= O\parentheses{x \sum_{p \leq \sqrt{x}} \ln p \sum_{n \geq 2} \frac{1}{p^n}}
    = O\parentheses{x \sum_{p \leq \sqrt{x}} \ln p \sum_{m \geq 0} \frac{1}{p^{m + 2}}} \\
    &= O\parentheses{x \sum_{p \leq \sqrt{x}} \frac{\ln p}{p^2} \sum_{m \geq 0} \frac{1}{p^m}}
    = O\parentheses{x \sum_{p \leq \sqrt{x}} \frac{\ln p}{p^2} \parentheses{\frac{1}{1 - 1/p}}} \\
    &= O\parentheses{x \sum_{p \leq \sqrt{x}} \frac{\ln p}{p^2} \parentheses{\frac{p}{p - 1}}}
    = O\parentheses{x \sum_{p \leq \sqrt{x}} \frac{\ln p}{p (p - 1)}}.
  \end{align*}
  Como seg\'un \Cref{lem:sum-lnp} la serie
  \(\displaystyle{\sum_{p = 2}^{\infty} \frac{\ln p}{p (p - 1)}}\) coverge,
  las sumas parciales est\'an acotadas,
  y reducimos a
  \[
    \sum_{n \leq x} \Psi\parentheses{\frac{x}{n}}\Lambda(n)
    - \sum_{p \leq x} \vartheta\parentheses{\frac{x}{p}} \ln p
    = O\parentheses{x \sum_{p = 2}^\infty \frac{\ln p}{p (p - 1)}}
    = O(x).
  \]
  
  Al inmiscuir \Cref{lem:psi-2xlnx-ox}, esta relaci\'on se troca por
  \[
    2x\ln x + O(x) - \Psi(x)\ln x
    - \sum_{p \leq x} \vartheta\parentheses{\frac{x}{p}}\ln p = O(x),
  \]
  o lo que es lo mismo por
  \[
    \Psi(x)\ln x + \sum_{p \leq x} \vartheta\parentheses{\frac{x}{p}}\ln p = 2x\ln x + O(x).
  \]
  De ac\'a, el uso consecutivo de
  \Cref{lem:psi-is-theta-osqrtxlnx} (para \(\Psi\))
  \[
    2x\ln x + O(x)
    = (\vartheta(x) + O(\sqrt{x}\ln x))\ln x +
    \sum_{p \leq x} \vartheta\parentheses{\frac{x}{p}}\ln p \\  
  \]
  y \Cref{lem:ln2x-is-sqrtx} (para \(\ln^2 x\))
  \begin{align*}
    \vartheta(x)\ln x + O(\sqrt{x}\ln^2 x) + \sum_{p \leq x} \vartheta\parentheses{\frac{x}{p}}\ln p
    &= \vartheta(x)\ln x + O(\sqrt{x}\sqrt{x}) + \sum_{p \leq x} \vartheta\parentheses{\frac{x}{p}}\ln p \\
    2x\ln x + O(x)
    &= \vartheta(x)\ln x + O(x) + \sum_{p \leq x} \vartheta\parentheses{\frac{x}{p}}\ln p.
  \end{align*}
  deviene en la secuencia.
\end{proof}

\begin{lemma}
  \label{lem:sum-1-n2}
  La serie \(\displaystyle{\sum_{n = 1}^\infty \frac{1}{n ^ 2}}\)
  converge a un n\'umero menor o igual a \(2\).
\end{lemma}

\begin{proof}
  Esto es sencillo si utilizamos sumas telesc\'opicas: 
  \[
    \sum_{n = 1}^k \frac{1}{n^2}
    = 1 + \sum_{n = 2}^k \frac{1}{n^2} 
    \leq 1 + \sum_{n = 2}^k \frac{1}{n (n - 1)}
    = 1 + \sum_{n = 2}^k \parentheses{\frac{1}{n - 1} - \frac{1}{n}}
    = 1 + \parentheses{\frac{1}{2 - 1} - \frac{1}{k}}
    = 2 - \frac{1}{k}.
  \]
  El resultado se sigue de inmediato.
\end{proof}

Finalmente el resultado te\'orico m\'as importante de este cap\'itulo.

\begin{theorem}
  [\cite{Selberg1949}]
  \label{thm:selberg}
  Para todo \(x \geq 1\) tenemos
  \[
    \sum_{p \leq x} \ln^2 p + \sum_{pq \leq x} \ln p \ln q = 2x\ln x + O(x).
  \]
\end{theorem}

\begin{note}
  En la f\'ormula dada arriba es indistinto si tomamos
  \(p\) y \(q\) distintos o si se permite que sean iguales.
  En efecto, la diferencia entre una y otra alternativa es apenas
  \[
    \sum_{p^2 \le x} (\ln p)^2
    \le \sum_{p^2 \le x} (\ln x^{1/2})^2
    \le \frac{\sqrt{x} \ln^2 x}{4}
    = O(x).
  \]
\end{note}

\begin{proof}[Prueba de la f\'ormula de Selberg]
  Consolidemos la diferencia en una \'unica suma 
  \begin{align*}
    \vartheta(x)\ln x - \sum_{p \leq x} \ln^2 p
    &= \sum_{p \leq x} \ln p \ln x - \sum_{p \leq x} \ln p \ln p \\
    &= \sum_{p \leq x} \ln p (\ln x - \ln p) \\
    &= \sum_{p \leq x} \ln p \ln \parentheses{\frac{x}{p}}.\eqlabel{eq:lnp-lnxp}
  \end{align*}
  
  A continuaci\'on recurrimos a una versi\'on gruesa de \Cref{lem:harmonic}:
  Al ser \(1/x\) acotado para \(x \ge 0\), obtenemos  para la serie arm\'onica
  \[
    \sum_{n \leq x} \frac{1}{n}
    = \ln x + \gamma + O\parentheses{\frac{1}{x}}
    = \ln x + O(1),
  \]
  o, lo que es lo mismo,
  \[
    \ln x = \sum_{n \leq x} \frac{1}{n} + O(1).
  \]
  
  Reeplazamos este nuevo estimado en \eqref{eq:lnp-lnxp} para conseguir
  \begin{align*}
    \vartheta(x)\ln x - \sum_{p \leq x} \ln^2 p
    &= \sum_{p \leq x} \ln p \ln \parentheses{\frac{x}{p}}
    = \sum_{p \leq x} \ln p \parentheses{\sum_{n \leq x/p} \frac{1}{n} + O(1)} \\
    &= \sum_{p \leq x} \ln p \sum_{n \leq x/p} \frac{1}{n} + O\parentheses{\sum_{p \leq x} \ln p}
    = \sum_{p \leq x} \sum_{n \leq x/p} \frac{\ln p}{n} + O\parentheses{\sum_{p \leq x} \ln p} \\
    &= \sum_{n \leq x} \sum_{p \leq x/n} \frac{\ln p}{n} + O\parentheses{\sum_{p \leq x} \ln p}
    = \sum_{n \leq x} \sum_{p \leq x/n} \frac{\ln p}{n} + O(\vartheta(x)).
    \eqlabel{eq:sum-sum-lnpn-o}
  \end{align*}
  Pero una combinaci\'on de \Cref{lem:psi-is-ox} y \Cref{lem:psi-is-theta-osqrtxlnx} conduce a
  \(\Psi(x)\) y \(\vartheta(x)\) est\'an en \(O(x)\),
  propiedad que utilizaremos en la forma \(O(\vartheta(x))=O(x)\).

  A continuaci\'on desdoblamos uno de los sumandos en \eqref{eq:sum-sum-lnpn-o} para llegar a 
  \begin{align*}
    \vartheta(x)\ln x - \sum_{p \leq x} \ln^2 p
    &= \sum_{n \leq x} \sum_{p \leq x/n} \frac{\ln p}{n} + O(x) \\
    &= \sum_{n \leq x} \frac{1}{n}\sum_{p \leq x/n} \ln p + O(x) \\
    &= \sum_{n \leq x} \frac{1}{n} \cdot \vartheta\parentheses{\frac{x}{n}} + O(x) \\
    &= O\parentheses{x \sum_{n \leq x} \frac{1}{n^2}} + O(x) \\
    &= O(2x) + O(x) \\
    &= O(x),
  \end{align*}
  pues la sumatoria de rec\'iprocos al cuadrado est\'a acotada por \(2\).
  
  Para el remate es cuesti\'on de reemplazar en \Cref{lem:theta-2xlnx-ox} y lograr  
  \begin{align*}
    2x\ln x + O(x)
    &= \vartheta(x)\ln x
    + \sum_{p \leq x} \vartheta\parentheses{\frac{x}{p}}\ln p  \\
    &= \sum_{p \leq x} \ln^2 p + O(x)
    + \sum_{p \leq x} \vartheta\parentheses{\frac{x}{p}}\ln p  \\
    &= \sum_{p \leq x} \ln^2 p
    + \sum_{p \leq x} \sum_{q \leq x/p} \ln q \ln p  \\
    &= \sum_{p \leq x} \ln^2 p + \sum_{pq \leq x} \ln p \ln q,
  \end{align*}
  la f\'ormula de Selberg.
\end{proof}
