\pretolerance=20000
\tolerance=30000
\setlength{\headheight}{14.61858pt}
\selectlanguage{spanish}
\pagestyle{fancy}

\chapter[La Identidad de Selberg]
{La Identidad de Selberg}
\label{Chapter03}

\begin{lemma}\label{lem:01}
Para todo \(x \geq 1\) tenemos
\[
  \sum_{n \leq x} \frac{1}{n} = \ln x + \gamma + O\left(\frac{1}{x}\right),
\]
aqu\'i \(\gamma \approx 0.52\) es una constante conocida como
\textbf{la constante de Euler}.
\end{lemma}

\begin{proof}
La funci\'on \(f : [1, \infty) \to \BR\) con \(f(x)=1/x\) es continua y
diferenciable en toda la recta conque podemos aplicar la f\'ormula de
sumaci\'on de Euler en cualquier intervalo \([2, k]\)
y as\'i obtener
\begin{align*}
  \sum_{n = 2}^k \frac{1}{n} &=
  \int_1^k \frac{dt}{t} + \int_{1}^k (t - \floor{t})\parentheses{\frac{1}{-t^2}} \, dt \\
  &= \ln k - \int_1^k \frac{t - \floor{t}}{t^2} \, dt,
\end{align*}
lo cual conduce de inmediato a 
\begin{align*}
  \sum_{n = 1}^k \frac{1}{n} - \ln k
  &= 1 - \int_1^k \frac{t - \floor{t}}{t^2} \, dt.
\end{align*}

Para analizar qu\'e ocurre cuando \(k \to \infty\) escribimos
\begin{align*}
  \gamma &= \lim_{k \to \infty} \parentheses{\sum_{n = 1}^k \frac{1}{n} - \ln k} \\
  &= 1- \lim_{k \to \infty}  \parentheses{\int_1^k \frac{t - \floor{t}}{t^2}} \, dt \\
  &= 1- \int_1^\infty \frac{t - \floor{t}}{t^2} \, dt, \eqlabel{eq:01}
\end{align*}
l\'imite que existe, pues al tenerse 
\begin{align*}
  \int_1^k \frac{t - \floor{t}}{t^2} \, dt
  \le \int_1^k \frac{1}{t^2} \, dt
  = 1 -\frac{1}{k}
  \le 1
\end{align*}
la convergencia queda garantizada por monotonicidad.

Finalmente, para establecer la f\'ormula anunciada
reemplazamos \eqref{eq:01} tras el uso de la f\'ormula de sumaci\'on de Euler
\begin{align*}
  \sum_{n \leq x} \frac{1}{n} &= \int_1^x \frac{1}{t} \, dt +
  \int_1^x \parentheses{t - \floor{t}} \parentheses{-\frac{1}{t^2}} \, dt
  - \parentheses{x - \floor{x}} \parentheses{\frac{1}{x}} + 1 \\
  &= \ln x - \int_1^x \frac{t - \floor{t}}{t^2} \, dt + 1 - \frac{x - \floor{x}}{x} \\
  &= \ln x - \int_1^x \frac{t - \floor{t}}{t^2} \, dt
  + 1 - \int_1^{\infty} \frac{t - \floor{t}}{t^2} \, dt +
  \int_1^{\infty} \frac{t - \floor{t}}{t^2} \, dt
  - \frac{x - \floor{x}}{x} \\
  &= \ln x + \gamma + \int_x^{\infty} \frac{t - \floor{t}}{t^2} \, dt
  - \frac{x - \floor{x}}{x} \\
  &\leq \ln x + \gamma + \int_x^{\infty} \frac{1}{t^2} \, dt - \frac{x - \floor{x}}{x} \\
  &= \ln x + \gamma + \frac{1 - \parentheses{x - \floor{x}}}{x}
\end{align*}
alcanzando nuestro objetivo.
\end{proof}

\begin{lemma}\label{lem:02}
  Para todo \(x \geq 1\) tenemos
  \[
    \sum_{n \leq x} \ln n = x\ln x - x + O\parentheses{\ln x}.
  \]
\end{lemma}

\begin{proof}
  Esta vez utilizamos la f\'ormula de sumaci\'on de Euler en
  \(f : [1, \infty) \to \BR\) con \(f(x) = \ln x\),
  continua y diferenciable en toda la recta real positiva:  
  \begin{align*}
    \sum_{n \leq x} \ln x &= \int_1^x \ln t \, dt +
    \int_1^x  \frac{t - \floor{t}}{t} \, dt + \parentheses{\floor{x} - x} \ln x  \\
    &= x \ln x - x + 1 + \int_1^x \frac{t - \floor{t}}{t} \, dt
    + \parentheses{\floor{x} - x} \ln x \\
    &\leq x \ln x - x + 1 + \int_1^x \frac{1}{t} \,dt + \ln x \\
    &= x \ln x - x + 1 + 2 \ln x.
  \end{align*}
  Como \(1\) est\'a dominado por \(\ln x\),
  se cumple \(\sum_{n \leq x} \ln n = x\ln x - x + O(\ln x)\), 
  lo anunciado.
\end{proof}

\begin{lemma}\label{lem:03}
  Para toda funci\'on aritm\'etica \(f\) se cumple
  \[
    \sum_{n \leq x} \sum_{d \mid n} f(d) =
    \sum_{n \leq x} f(n) \floor{\frac{x}{n}}.
  \]
\end{lemma}

\begin{proof}
  Sean \(f\) y \(g\) dos funciones aritm\'eticas,
  \(F\) y \(G\) sus respectivas cumulativas;
  es decir, \(F(x) = \sum_{n \leq x} f(x)\) y
  \(G(x) = \sum_{n \leq x} g(x)\).
  La cumulativa del producto de Dirichlet de \(f\) y \(g\) est\'a dada por 
  \begin{align*}
    \sum_{n \leq x} f * g (n) &= \sum_{n \leq x} \sum_{cd = n} f(c)g(d) \\
    &= \sum_{c \leq x} \sum_{d \leq \frac{x}{c}} f(c) g(d) \\
    &= \sum_{c \leq x} f(c) \sum_{d \leq \frac{x}{c}} g(d) \\
    &= \sum_{c \leq x} f(c) G\parentheses{\frac{x}{c}}.\eqlabel{eq:02}
  \end{align*}

  En particular, cuando \(g = \mathbf{1}\), su cumulativa es
  \[
    \sum_{n \leq x} \mathbf{1}(n) = \sum_{n \leq x} 1 = \floor{x}.
  \]
  De este modo, al introducir \(G(x) = \floor{x}\) en \eqref{eq:02} se logra
  \[
    \sum_{n \leq x} \sum_{d \mid n} f(d)
    = \sum_{n \leq x} f * \mathbf{1} (n)
    = \sum_{n \leq x} f(n) \floor{\frac{x}{n}},
  \]
  lo buscado.
\end{proof}

\begin{lemma}\label{lem:04}
  Para todo n\'umero real \(x\) tenemos
  \[
    \floor{x} = x + O(1).
  \]
\end{lemma}

\begin{proof}
  Sea \(x = n + r\) un n\'umero real no negativo, con
  \(n\) entero y \(0 \leq r < 1\).
  De esta manera, por definici\'on de m\'aximo entero obtenemos
  \[
    \floor{x} = n = x - r = x + O(1)
  \]
  concluyendo trivialmente con la propiedad.
\end{proof}

\begin{lemma}\label{lem:05}
  Para todo \(x \geq 1\) tenemos
  \[
    \Psi(x) = O(x).
  \]
\end{lemma}

\begin{proof}
  Usaremos el desarrollo del teorema de Chebyshev por Diamond \cite{Diamond1982}
  \[
    A \leq \liminf \frac{\Psi(x)}{x} \leq \limsup \frac{\Psi(x)}{x} \leq \frac{6A}{5}
  \]
  con
  \[
    A = -\frac{\ln 1}{1} + \frac{\ln 2}{2} + \frac{\ln 3}{3} + \frac{\ln 5}{5} - \frac{\ln 30}{30}
    \approx 0.92129202293409078091340844996160...
  \]
  Reescribimos la parte derecha de la desigualdad con valor absoluto ya que es una funci\'on positiva cual
  \[
    \limsup \frac{\abs{\Psi(x)}}{x} \leq \frac{6A}{5}.
  \]
  Por definici\'on existe \(n_0\) a partir del cual se tiene 
  \[
    \frac{\abs{\Psi(x)}}{x} \leq \frac{6A}{5},
  \]
  es decir, para todo \(x \geq n_0\).
  Como ello equivale a
  \[
    \abs{\Psi(x)} \leq \parentheses{\frac{6A}{5}} x,
  \]
  se consigue \(\Psi(x) = O(x)\).
\end{proof}

\begin{lemma}\label{lem13}
La funci\'on de Mangoldt se puede expresar como el siguiente producto de Dirichlet
$$\Lambda = \mu * \ln.$$
\end{lemma}
\begin{proof}
Esta f\'ormula equivale a  $\Lambda * 1 = \ln$ v\'ia  inversi\'on de M\"obius.  
\end{proof}

\begin{lemma}\label{lem14}
Para todo $x \geq 1$ tenemos
\[
\sum_{n \leq x} \frac{\Lambda(n)}{n} = \ln x + O(1).
\]
\end{lemma}

\begin{proof}
El desarrollo de  $\ln = \Lambda * 1$ cual 
$\ln n = \sum_{d \mid n} \Lambda(d)$ 
lleva a
\begin{align}
\sum_{n \leq x} \ln n = \sum_{n \leq x} \sum_{d \mid n} \Lambda(d).
\end{align}
Una aplicaci\'on directa del \cref{lem10} deriva en 
\begin{align}
\sum_{n \leq x} \ln n = \sum_{n \leq x} \Lambda(n) \floor*{\frac{x}{n}}. 
\end{align}
De ac\'a, en uso del \cref{lem11} conseguimos
\begin{align}
\sum_{n \leq x} \ln n &= \sum_{n \leq x} \Lambda(n) \left(\frac{x}{n} + O(1)\right) \\
&= x \sum_{n \leq x} \frac{\Lambda(n)}{n} + O\left(\sum_{n \leq x} \Lambda(x) \right) \\
&= x \sum_{n \leq x} \frac{\Lambda(n)}{n} + O(\Psi(x)) \\
&= x \sum_{n \leq x} \frac{\Lambda(n)}{n} + O(x)
\end{align}
dado que, por el \cref{lem12}, $\Psi(x) = O(x)$ claramente implica $O(\Psi(x))=O(x)$. 
Si aplicamos el \cref{lem09} al lado izquierdo desembocamos en
\begin{align}
x\ln x - x + O(\ln x) = x \sum_{n \leq x} \frac{\Lambda(n)}{n} + O(x). 
\end{align}
Al despejar obtenemos 
\begin{align}
\sum_{n \leq x} \frac{\Lambda(n)}{n} &= \frac{x \ln x}{x} - 1 + O\left(\frac{\ln x}{x}\right) + O(1) \\
&= \ln x - 1 + O(1) + O\left(\frac{\ln x}{x}\right) \\
%&\leq \ln x - 1 + c_0 + O(1) \\
%&\leq \ln x + c_1 \\
&= \ln x + O(1),
\end{align}
pues $-1 + O(1)+ O(\ln x/x)$ es acotado. 
\end{proof}

\begin{lemma}\label{lem15}
Para $f, g : [1, \infty) \to \BR$ sujetos a  $g(x) = \sum_{n \leq x} f\left(\frac{x}{n}\right) \ln x$ tenemos
\[
\sum_{n \leq x} \mu(n) g\left(\frac{x}{n}\right) = f(x) \ln(x) + \sum_{n \leq x} f\left(\frac{x}{n}\right) \Lambda(n).
\]
\end{lemma}
\begin{proof}
Desarrollemos la sumatoria que queremos analizar
\begin{align}
\sum_{n \leq x} \mu(n) g\left(\frac{x}{n}\right) &= \sum_{n \leq x} \mu(n) \sum_{m \leq x/n} f\left(\frac{x}{nm}\right) \ln \left(\frac{x}{n}\right) \\
&= \sum_{nm \leq x} \mu(n) \ln\left(\frac{x}{n}\right) f \left(\frac{x}{nm}\right) \\    
&= \sum_{c \leq x} f\left(\frac{x}{c}\right) \sum_{d \mid c} \mu(d) \ln\left(\frac{x}{d}\right) \\
&= \sum_{n \leq x} f\left(\frac{x}{n}\right) \sum_{d \mid n} \mu(d) \left[\ln \left(\frac{x}{n}\right) + \ln\left(\frac{n}{d}\right)\right] \\
&= \left[\sum_{n \leq x} f\left(\frac{x}{n}\right) \ln\left(\frac{x}{n}\right) \sum_{d \mid n} \mu(d)\right]
+  \left[\sum_{n \leq x} f\left(\frac{x}{n}\right) \sum_{d \mid n} \mu(d) \ln\left(\frac{n}{d}\right)\right].\\
&= f(x) +  \ln{x} +  \sum_{n \leq x} f\left(\frac{x}{n}\right) (\mu * \ln)(n). 
\end{align}
Con ello, finalmente, 
utilizamos el \cref{lem13} para concluir lo deseado. 
\end{proof}

\begin{lemma}\label{lem16}
Para todo $x \geq 1$ tenemos
\[
\ln^2 x = O(\sqrt{x}).
\]
\end{lemma}

\begin{proof}
Como sabemos que para todo $x \geq 1$ se cumple que $x > \ln x$ (v\'ia an\'alisis de la  derivada de $x - \ln x$), 
se obtiene
\begin{align}
\ln^2 x &= \ln^2 ((x ^ {\frac{1}{4}}) ^ 4) \\
&= 16 \ln^2 (x ^ {\frac{1}{4}}) \\
&< 16 (x ^ {\frac{1}{4}}) ^ 2 \\
&= 16 \sqrt{x}.
\end{align}
\end{proof}

\begin{lemma}\label{lem17}
Para todo $x \geq 1$ tenemos
$$\Psi(x) \ln x + \sum_{n \leq x} \Psi\left(\frac{x}{n}\right)\Lambda(n) = 2x\ln x + O(x).$$
\end{lemma}

\begin{proof}
Para utilizar el \cref{lem15}, definimos convenientemente $f : [1, \infty) \to \BR$ con
\begin{align}
f(x) = \Psi(x) - x + \gamma + 1.
\end{align}
Antes de aplicar el \cref{lem15}, le brindaremos a $g(x) = \sum_{n \leq x} f\left(\frac{x}{n}\right)\ln x$ 
una expansi\'on diferente cual es 
\begin{align}
g(x)= \sum_{n \leq x} f\left(\frac{x}{n}\right)\ln x &= \sum_{n \leq x} \left(\Psi\left(\frac{x}{n}\right) - \frac{x}{n} + \gamma + 1\right) \ln x \\
&= \sum_{n \leq x} \Psi\left(\frac{x}{n}\right) \ln x - x \ln x \sum_{n \leq x} \frac{1}{n} + (\gamma + 1) \ln x \sum_{n \leq x} 1.\label{eq50}
\end{align}
Analicemos por separado cada sumatoria de la Ecuaci\'on \ref{eq50}.

La primera resulta ser 
\begin{align}
\sum_{n \leq x} \Psi\left(\frac{x}{n}\right) = \sum_{n \leq x} \sum_{d \leq \frac{x}{n}} \Lambda(d) = 
 \sum_{n \leq x} \sum_{d \mid n} \Lambda(d) = 
 \sum_{n \leq x} (\Lambda * 1)(n) = 
\sum_{n \leq x} \ln n, 
\end{align}
de tipo $x\ln x - x + O(\ln x)$ por el \cref{lem09}.
Al multiplicar el logaritmo obtenemos la expresi\'on
\begin{align}
\sum_{n \leq x} \Psi\left(\frac{x}{n}\right)\ln x = x \ln^2 x - x \ln x + O(\ln^2 x).
\end{align}

Para la segunda recurrimos al \cref{lem08} y logramos 
\begin{align}
-x \ln x \sum_{n \leq x} \frac{1}{n} &= -x \ln x \left(\ln x + \gamma + O\left(\frac{1}{x}\right)\right) \\
&= -x \ln^2 x - \gamma x \ln x + O(\ln x).
\end{align}

Para la tercera necesitamos el \cref{lem11}: 
\begin{align}
(\gamma + 1) \ln x \sum_{n \leq x} 1 &= (\gamma + 1) \ln x \floor{x} \\
&= (\gamma + 1) \ln x (x + O(1)) \\
&= (\gamma + 1) x \ln x + O(\ln x).
\end{align}

Finalmente, juntamos los tres resultados y obtenemos
\begin{align}
g(x) &= x \ln^2 x - x \ln x + O(\ln^2 x) - x \ln^2 x - \gamma x \ln x + O(\ln x) + (\gamma + 1) x \ln x + O(\ln x) \\
&= O(\ln^2 x) + O(\ln x) \\
&= O(\ln^2 x). 
\end{align}

Del \cref{lem15} obtenemos entonces 
\begin{align}
\sum_{n \leq x} \mu(n) g\left(\frac{x}{n}\right) = 
(\Psi(x) - x + \gamma + 1) \ln x + \sum_{n \leq x} \left(\Psi\left(\frac{x}{n}\right) - \frac{x}{n} + \gamma + 1\right) \Lambda(n). 
\end{align}
El remate consiste en analizar ambos miembros de la desigualdad por separado. 

Utilizamos la desigualdad triangular el hecho de que se cumple $g(x) = O(\ln^2 x)$ para obtener a la izquierda 
\begin{align}
\left | \sum_{n \leq x} \mu(n) g\left(\frac{x}{n}\right) \right|  \leq \sum_{n \leq x} \left| g\left(\frac{x}{n}\right) \right| 
= O\left(\sum_{n \leq x} g\left(\frac{x}{n}\right)\right) 
= O\left(\sum_{n \leq x} \ln^2 \left(\frac{x}{n}\right)\right).
\end{align}
Con esta expresi\'on el \cref{lem16} permite conseguir
\begin{align}
\sum_{n \leq x} \mu(n) g\left(\frac{x}{n}\right) = O\left(\sum_{n \leq x} \sqrt{\frac{x}{n}}\right) 
= O\left(\sqrt{x} \sum_{n \leq x} \frac{1}{\sqrt{n}}\right) = O\left( \sqrt{x}\cdot \sqrt{x}\right)=O(x). 
\end{align}

El t\'ermino de la derecha lo reordenamos cual 
\begin{align}
 \Psi(x)\ln x + \sum_{n \leq x} \Psi\left(\frac{x}{n}\right)\Lambda(n) - x \ln x - x \sum_{n \leq x} \frac{\Lambda(n)}{n} + (\gamma + 1)\Psi(x).\label{eq64}
\end{align}
Merced a \cref{lem12} y \cref{lem14} reducimos la Expresi\'on \ref{eq64}
\begin{align}
& \hskip -0.5in  \Psi(x)\ln x + \sum_{n \leq x} \Psi\left(\frac{x}{n}\right)\Lambda(n) - x \ln x - x(\ln x + O(1)) + O(x)\\
&= \Psi(x)\ln x + \sum_{n \leq x} \Psi\left(\frac{x}{n}\right)\Lambda(n) - 2x \ln x + O(x).
\end{align}
Finalmente, igualamos los resultados de ambas partes
\begin{align}
\Psi(x)\ln x + \sum_{n \leq x} \Psi\left(\frac{x}{n}\right)\Lambda(n) - 2x \ln x + O(x) = O(x), 
\end{align}
equivalente a lo aseverado. 
\end{proof}

\begin{lemma}\label{lem18}
Para todo $x \geq 1$ tenemos
\[
\Psi(x) = \vartheta(x) + O(\sqrt{x}\ln x).
\]
\end{lemma}

\begin{proof}
Directo de la definici\'on observamos que se cumple 
\begin{align}
\Psi(x) = \sum_{n = 1}^\infty \vartheta(x^\frac{1}{n}).
\end{align}
Notemos que al mismo tiempo esta sumatoria tiene apenas una cantidad finita de t\'erminos efectivos puesto que la funci\'on $\vartheta$ 
solo tiene sentido cuando es evaluada en valores mayores o iguales a $2$. 
Para un $x$ espec\'ifico,  hallamos ese momento $m=m(x)$ mediante la cadena 
\begin{align}
x^\frac{1}{m} &\geq 2 \\
x^\frac{2}{m} &\geq 4 \\
x^\frac{2}{m} &> e \\
\frac{2}{m}\ln x &> \ln e \\
\frac{2}{m}\ln x &> 1 \\
m &< 2 \ln x
\end{align}
y notamos que para valores mayores $m = \floor{2 \ln x}$ los constituyentes de la suma son nulos.
Ahora podemos escribir a $\Psi$ como 
\begin{align}
\Psi(x) &= \vartheta(x) + \vartheta(x^\frac{1}{2}) + \cdots + \vartheta(x^\frac{1}{m}) \\
\Psi(x) &= \vartheta(x) + \sum_{n = 2}^m \vartheta(x^\frac{1}{n}).
\end{align}

Para el an\'alisis de $\sum_{n = 2}^m \vartheta(x^\frac{1}{n})$ desdoblamos 
\begin{align}
\sum_{n = 2}^m \vartheta(x^\frac{1}{n}) = \sum_{n = 2}^m \sum_{p \leq x^\frac{1}{n}} \ln p. 
\end{align}
Trataremos de darle forma manipulativa sencilla. 
Si un n\'umero primo $p$ ser\'a inmiscuido en la sumatoria, 
su logaritmo contribuir\'a a la sumatoria tantas veces como las ra\'ices de $x$ lo permitan: 
entre $1$ y $k$, donde $k$ es el m\'aximo entero que obedece $p^k \le x$.  
F\'acilmente hallamos que este m\'aximo est\'a dado por 
\begin{align}
k = \floor*{\frac{\ln x}{\ln p}}.
\end{align}

Por su parte, para forzar por lo menos $p^2 \le x$, se necesita $p \le \sqrt{x}$,
detalle importante que aprovecharemos.

Ahora analicemos la forma equivalente de la doble sumatoria
\begin{align}
\sum_{n = 2}^m \sum_{p^n \leq x} \ln p &= \sum_{p \leq \sqrt{x}} \, \, \sum_{2 \leq n \leq \floor*{\frac{\ln x}{\ln p}}} \ln p \\
&= \sum_{p \leq \sqrt{x}} \ln p \sum_{2 \leq n \leq \floor*{\frac{\ln x}{\ln p}}} 1 \\
&\leq \sum_{p \leq \sqrt{x}} \ln p \floor*{\frac{\ln x}{\ln p}} \\
&\leq \sum_{p \leq \sqrt{x}} \ln p \left(\frac{\ln x}{\ln p}\right) \\
&= \sum_{p \leq \sqrt{x}} \ln x \\
&= \ln x \sum_{p \leq \sqrt{x}} 1 \\
&\leq \ln x \sum_{n \leq \sqrt{x}} 1 \\
&\leq \ln x \sqrt{x}, 
\end{align}
lo que permite concluir $\sum_{n = 2}^m \vartheta(x^\frac{1}{n}) = O(\sqrt{x} \ln x)$. 

Con lo anterior queda establecida la relaci\'on $\Psi(x) = \vartheta(x) + O(\sqrt{x} \ln x)$.
\end{proof}

\begin{lemma}\label{lem19}
La serie
\[
\sum_{p = 2}^{\infty} \frac{\ln p}{p (p - 1)}
\]
tomada sobre los primos converge.
\end{lemma}

\begin{proof}
El primer paso es notar que el l\'imite
\begin{align}
\lim_{n \to \infty} \frac{\ln n}{n (n - 1)} n^\frac{3}{2}
\end{align}
vale $0$ como se deduce al descomponer 
\begin{align}
\frac{\ln n}{n (n - 1)} n^\frac{3}{2} = \left(\frac{\ln n}{\sqrt{n}}\right)\left(\frac{n}{n - 1}\right).
\end{align}

Analicemos el l\'imite de lo que est\'a dentro del par\'entesis de la izquierda
\begin{align}
\lim_{n \to \infty} \frac{\ln n}{\sqrt{n}} = \frac{\infty}{\infty}.
\end{align}
Como este es de la forma $\frac{\infty}{\infty}$, aplicamos la regla de L'Hospital
\begin{align}
\lim_{n \to \infty} \frac{\ln n}{\sqrt{n}}
= \lim_{n \to \infty}\frac{\frac{1}{n}}{\frac{1}{2\sqrt{n}}}
= \lim_{n \to \infty} \frac{2}{\sqrt{n}} = 0.
\end{align}

Ahora analicemos el l\'imite de lo que est\'a dentro del par\'entesis de la derecha
\begin{align}
\lim_{n \to \infty} \frac{n}{n - 1}
= \lim_{n \to \infty} \frac{1}{1 - \frac{1}{n}}
= \frac{1}{1 - 0} = 1.
\end{align}

Como ambos l\'imites existen, por aritm\'etica de l\'imites obtenemos
\begin{align}
\lim_{n \to \infty} \frac{\ln n}{n (n - 1)} n^\frac{3}{2}
= \left(\lim_{n \to \infty} \frac{\ln n}{\sqrt{n}}\right)\left(\lim_{n \to \infty} \frac{n}{n - 1}\right)
= 0 \cdot 1 = 0.
\end{align}

Por supuesto, lo mismo es v\'alido si crecemos a lo largo de primos, conque se tiene 
\begin{align}
\lim_{p \to \infty} \frac{\ln p}{p (p - 1)} p^\frac{3}{2}=0.
\end{align}

Por definici\'on entonces, dado $\epsilon > 0$, existe un $n_0$ 
a partir del cual se tiene 
\begin{align}
\frac{\ln p}{p (p - 1)} p^\frac{3}{2} < \epsilon
\end{align}

De este modo, al hacer $n_1 = \floor{n_0} + 1$, sabemos que para todo $p \geq n_1$ se tendr\'a 
\begin{align}
\sum_{p \geq n_1} \frac{\ln p}{p (p - 1)} <  \sum_{p \geq n_1} \frac{\epsilon}{p^\frac{3}{2}}. 
\end{align}

Como el menor primo es 2, logramos el estimado 
\begin{align}
\sum_{p \geq n_1} \frac{\ln p}{p (p - 1)} <  \sum_{p \geq n_1} \frac{\epsilon}{p^\frac{3}{2}} \le \int_{1}^\infty \frac{\epsilon}{x^\frac{3}{2}} dx  < 2 \epsilon. 
\end{align}

Esto, por supuesto, lleva a 
\begin{align}
\sum_{p = 2}^\infty \frac{\ln p}{p (p - 1)} < \sum_{p < n_1} \frac{\ln p}{p (p - 1)} + 2 \epsilon, 
\end{align}
lo que equivale a la convergencia absoluta de la serie. 
\end{proof}

\begin{lemma}\label{lem20}
Para todo $x \geq 1$ tenemos
\[
\vartheta(x)\ln x + \sum_{p \leq x} \vartheta\left(\frac{x}{p}\right)\ln p = 2x\ln x + O(x).
\]
\end{lemma}

\begin{proof}
El primer paso es comparar la sumatoria con otra m\'as a tono con nuestros intereses: 
\begin{align}
\sum_{n \leq x} \Psi\left(\frac{x}{n}\right)\Lambda(n) - \sum_{p \leq x} \vartheta\left(\frac{x}{p}\right)\ln p
&= \sum_{n \leq x} \sum_{m \leq \frac{x}{n}} \Lambda(m)\Lambda(n) - \sum_{p \leq x} \sum_{q \leq \frac{x}{p}} \ln q \ln p \\
&= \sum_{nm \leq x} \Lambda(n)\Lambda(m) - \sum_{pq \leq x} \ln p \ln q.
\end{align}
En el primero de los dos sumandos sobrevivientes, la funci\'on de Mangoldt solo act\'ua sobre las potencias de los primos. 
En particular, todas las combinaciones de primos con potencias iguales a uno van de la mano con la sumatoria que estamos restando a la derecha.
De esta manera, apenas sobreviven aquellos t\'erminos 
con al menos uno de los exponentes mayor o igual a dos. 
De este modo, se consigue 
\begin{align}
\sum_{n \leq x} \Psi\left(\frac{x}{n}\right)\Lambda(n) - \sum_{p \leq x} \vartheta\left(\frac{x}{p}\right)\ln p & \le 
\sum_{\substack{p^n q^m \leq x \\ n \geq 2, m \geq 1}} \ln p \ln q  + 
\sum_{\substack{p^n q^m \leq x \\ m \geq 2, n \geq 1}} \ln p \ln q \\
& = 2 \sum_{\substack{p^n q^m \leq x \\ n \geq 2, m \geq 1}} \ln p \ln q \\
& = 2 \sum_{\substack{p^n \leq x \\ n \geq 2}} \ln p \sum_{\substack{q^m \leq \frac{x}{p^n} \\ m \geq 1}} \ln q  \\
& = O\left(\sum_{\substack{p^n \leq x \\ n \geq 2}} \ln p \, \Psi\left(\frac{x}{p^n}\right)\right), 
\end{align}
puesto que tras desigualdad contamos por partida doble aquellos pares con ambos exponentes al menos dos y ello contribuyen con valores positivos. 
%Debido a que los primos deben tomarse por pares,
%podemos asegurarnos de este exponente en el primer n\'umero primo y que el exponente del segundo n\'umero primo tome cualquier valor
%\begin{align*}
%\sum_{n \leq x} \Psi\left(\frac{x}{n}\right)\Lambda(n) - \sum_{p \leq x} \vartheta\left(\frac{x}{p}\right)\ln p
%&= O\left(\sum_{\substack{p^n q^m \leq x \\ n \geq 2, m \geq 1}} \ln p \ln q\right) \\
%&= O\left(\sum_{\substack{p^n \leq x \\ n \geq 2}} \ln p \sum_{\substack{q^m \leq \frac{x}{p^n} \\ m \geq 1}} \ln q\right) \\
%&= O\left(\sum_{\substack{p^n \leq x \\ n \geq 2}} \ln p \, \Psi\left(\frac{x}{p^n}\right)\right).
%\end{align*}
Para continuar, utilizamos el \cref{lem12} en la \'ultima igualdad y logramos 
\begin{align}
\sum_{n \leq x} \Psi\left(\frac{x}{n}\right)\Lambda(n) - \sum_{p \leq x} \vartheta\left(\frac{x}{p}\right)\ln p
&= O\left(\sum_{\substack{p^n \leq x \\ n \geq 2}} \ln p \, \frac{x}{p^n}\right) \\
&= O\left(x\sum_{\substack{p^n \leq x \\ n \geq 2}} \frac{\ln p}{p^n}\right).
\end{align}

Llegado este punto, nuevamente notamos que la contribuci\'on en la cola es exclusiva de los primos sujetos a $p^2 \le x$. 
Asimismo, los t\'erminos de la sumatoria est\'an dominados por una serie geom\'etrica, 
por lo que conseguimos 
\begin{align}
\sum_{n \leq x} \Psi\left(\frac{x}{n}\right)\Lambda(n) - \sum_{p \leq x} \vartheta\left(\frac{x}{p}\right)\ln p
&= O\left(x \sum_{p \leq \sqrt{x}} \ln p \sum_{n \geq 2} \frac{1}{p^n}\right) \\
%
%\sum_{n \leq x} \Psi\left(\frac{x}{n}\right)\Lambda(n) - \sum_{p \leq x} \vartheta\left(\frac{x}{p}\right)\ln p
&= O\left(x \sum_{p \leq \sqrt{x}} \ln p \sum_{m \geq 0} \frac{1}{p^{m + 2}}\right) \\
&= O\left(x \sum_{p \leq \sqrt{x}} \frac{\ln p}{p^2} \sum_{m \geq 0} \frac{1}{p^m}\right) \\
&= O\left(x \sum_{p \leq \sqrt{x}} \frac{\ln p}{p^2} \left(\frac{1}{1 - \frac{1}{p}}\right)\right) \\
&= O\left(x \sum_{p \leq \sqrt{x}} \frac{\ln p}{p^2} \left(\frac{p}{p - 1}\right)\right) \\
&= O\left(x \sum_{p \leq \sqrt{x}} \frac{\ln p}{p (p - 1)}\right).
\end{align}
Como seg\'un el \cref{lem19} la serie $\displaystyle{\sum_{p = 2}^{\infty} \frac{\ln p}{p (p - 1)}}$ coverge, 
las sumas parciales est\'an acotadas, y reducimos a 
\begin{align}
\sum_{n \leq x} \Psi\left(\frac{x}{n}\right)\Lambda(n) - \sum_{p \leq x} \vartheta\left(\frac{x}{p}\right)\ln p
&= O\left(x \sum_{p = 2}^\infty \frac{\ln p}{p (p - 1)}\right) \\
&= O(x).
\end{align}

Al inmiscuir al \cref{lem17}, esta relaci\'on se troca por 
\begin{align}
2x\ln x + O(x) - \Psi(x)\ln x - \sum_{p \leq x} \vartheta\left(\frac{x}{p}\right)\ln p = O(x), 
\end{align}
o lo que es lo mismo por 
\begin{align}
\Psi(x)\ln x + \sum_{p \leq x} \vartheta\left(\frac{x}{p}\right)\ln p = 2x\ln x + O(x).
\end{align}
De ac\'a, el uso consecutivo del \cref{lem18} (para $\Psi$) y el \cref{lem16} (para $\ln^2 x$) deviene en la secuencia
\begin{align}
(\vartheta(x) + O(\sqrt{x}\ln x))\ln x + \sum_{p \leq x} \vartheta\left(\frac{x}{p}\right)\ln p &= 2x\ln x + O(x), \\
\vartheta(x)\ln x + O(\sqrt{x}\ln^2 x) + \sum_{p \leq x} \vartheta\left(\frac{x}{p}\right)\ln p &= 2x\ln x + O(x), \\
\vartheta(x)\ln x + O(\sqrt{x}\sqrt{x}) + \sum_{p \leq x} \vartheta\left(\frac{x}{p}\right)\ln p &= 2x\ln x + O(x), \\
\vartheta(x)\ln x + O(x) + \sum_{p \leq x} \vartheta\left(\frac{x}{p}\right)\ln p &= 2x\ln x + O(x).
\end{align}
%Concluimos que se cumple $\vartheta(x)\ln x + \sum_{p \leq x} \vartheta\left(\frac{x}{p}\right)\ln p = 2x\ln x + O(x)$ para todo $x \geq 1$.
\end{proof}

\begin{lemma}
La serie
\[
\sum_{n = 1}^\infty \frac{1}{n ^ 2}
\]
converge a un n\'umero menor o igual a $2$. 
\end{lemma}

\begin{proof}
Esto es sencillo si utilizamos sumas telesc\'opicas: 
\begin{align}
\sum_{n = 1}^k \frac{1}{n^2} &= 1 + \sum_{n = 2}^k \frac{1}{n^2} \\
&\leq 1 + \sum_{n = 2}^k \frac{1}{n (n - 1)} \\
&= 1 + \sum_{n = 2}^k \left(\frac{1}{n - 1} - \frac{1}{n}\right) \\
&= 1 + \left(\frac{1}{2 - 1} - \frac{1}{k}\right) \\
&= 2 - \frac{1}{k}.
\end{align}
El resultado se sigue de inmediato. 
\end{proof}

Finalmente el resultado te\'orico m\'as importante de esta recopilaci\'on. 

\begin{theorem}[F\'ormula asint\'otica de Selberg]\label{the22}
Para todo $x \geq 1$ tenemos
\[
\sum_{p \leq x} \ln^2 p + \sum_{pq \leq x} \ln p \ln q = 2x\ln x + O(x).
\]
\end{theorem}

\begin{note}
En la f\'ormula dada arriba es indistinto si toman $p,q$ distintos o si se permite que sean iguales. 
En efecto, la diferencia entre una y otra alternativa es apenas  
\[
\sum_{p^2 \le x} (\ln p)^2 \le \sum_{p^2 \le x} (\ln x^{1/2})^2 \le \frac{\sqrt{x} \ln^2 x}{4}= O(x).  
\]
\end{note}

\begin{proof}[Prueba de la f\'ormula de Selberg]
Consolidemos la diferencia en una \'unica suma 
\begin{align}
\vartheta(x)\ln x - \sum_{p \leq x} \ln^2 p &= \sum_{p \leq x} \ln p \ln x - \sum_{p \leq x} \ln p \ln p \\
&= \sum_{p \leq x} \ln p (\ln x - \ln p) \\
&= \sum_{p \leq x} \ln p \ln \left(\frac{x}{p}\right).\label{eq127}
\end{align}

A continuaci\'on recurrimos a una versi\'on gruesa del \cref{lem08}:
Al ser $\dfrac{1}{x}$ acotado para $x \ge 0$, obtenemos  para la serie arm\'onica 
\begin{align}
\sum_{n \leq x} \frac{1}{n} &= \ln x + \gamma + O\left(\frac{1}{x}\right) \\
&= \ln x + O(1),  
\end{align}
o, lo que es lo mismo, 
\begin{align}
\ln x = \sum_{n \leq x} \frac{1}{n} + O(1). 
\end{align}

Reeplazamos este nuevo estimado en la Ecuaci\'on \ref{eq127} para conseguir 
\begin{align}
\vartheta(x)\ln x - \sum_{p \leq x} \ln^2 p &= \sum_{p \leq x} \ln p \ln \left(\frac{x}{p}\right) \\
&= \sum_{p \leq x} \ln p \left(\sum_{n \leq \frac{x}{p}} \frac{1}{n} + O(1)\right) \\
&= \sum_{p \leq x} \ln p \sum_{n \leq \frac{x}{p}} \frac{1}{n} + O\left(\sum_{p \leq x} \ln p\right) \\
&= \sum_{p \leq x} \sum_{n \leq \frac{x}{p}} \frac{\ln p}{n} + O\left(\sum_{p \leq x} \ln p\right) \\
&= \sum_{n \leq x} \sum_{p \leq \frac{x}{n}} \frac{\ln p}{n} + O\left(\sum_{p \leq x} \ln p\right) \\
&= \sum_{n \leq x} \sum_{p \leq \frac{x}{n}} \frac{\ln p}{n} + O(\vartheta(x)).\label{eq136}
\end{align}
Pero una combinaci\'on de \cref{lem12} y \cref{lem18} conduce a 
\begin{align}
\Psi(x) &= O(x),  \\
\vartheta(x) &= O(x),
\end{align}
propiedad que utilizaremos en la forma $O(\vartheta(x))=O(x)$.

A continuaci\'on desdoblamos uno de los sumandos en Ecuaci\'on \ref{eq136} para llegar a 
\begin{align}
\vartheta(x)\ln x - \sum_{p \leq x} \ln^2 p &= \sum_{n \leq x} \sum_{p \leq \frac{x}{n}} \frac{\ln p}{n} + O(x) \\
&= \sum_{n \leq x} \frac{1}{n}\sum_{p \leq \frac{x}{n}} \ln p + O(x) \\
&= \sum_{n \leq x} \frac{1}{n}\cdot \vartheta\left(\frac{x}{n}\right) + O(x) \\
%&= \sum_{n \leq x} \frac{O\left(\frac{x}{n}\right)}{n} + O(x) \\
&= O\left(x \sum_{n \leq x} \frac{1}{n^2}\right) + O(x) \\
&= O(2x)+O(x)=O(x), 
\end{align}
pues la sumatoria de rec\'iprocos al cuadrado est\'a acotada por $2$. 

Para el remate es cuesti\'on de reemplazar en el \cref{lem20} y lograr  
\begin{align}
2x\ln x + O(x) &= \vartheta(x)\ln x + \sum_{p \leq x} \vartheta\left(\frac{x}{p}\right)\ln p  \\
&= \sum_{p \leq x} \ln^2 p + O(x) + \sum_{p \leq x} \vartheta\left(\frac{x}{p}\right)\ln p  \\
&= \sum_{p \leq x} \ln^2 p + \sum_{p \leq x} \sum_{q \leq \frac{x}{p}} \ln q \ln p  \\
&= \sum_{p \leq x} \ln^2 p + \sum_{pq \leq x} \ln p \ln q, 
\end{align}
la f\'ormula de Selberg.
\end{proof}
