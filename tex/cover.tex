\thispagestyle{empty}

\bigskip

\begin{center}
  {\baselineskip=30pt \Large Pontificia Universidad Cat\'olica del Per\'u} \\
  {\baselineskip=30pt \large Facultad de Ciencias e Ingenier\'ia}
\end{center}

\bigskip

\begin{figure}[H]
  \begin{center}
    \includegraphics[width=9.5cm]{images/2021-pucp-logo.png}
  \end{center}
\end{figure}

\begin{center}
  \begin{minipage}{14.0cm}
    \begin{center}
      % \textcolor{pucp}
      {\textbf{\huge{An\'alisis, algoritmos y estimados de la identidad de Selberg}}}
      \index{Car\'atula}
    \end{center}
  \end{minipage}
\end{center}

\vspace*{1.00cm}

\begin{center}
  Trabajo de investigaci\'on para obtener el grado de \\
  Bachiller en Ciencias con menci\'on en Matem\'atica
\end{center}

\vspace*{1.00cm}

\begin{center}
  Autor \\
  \textbf{\textsc{Manuel Alejandro Loaiza Vasquez}}
  % DNI \texttt{70452019}
\end{center}

\vspace*{0.5cm}

\begin{center}
  Asesor \\
  \textbf{\textsc{Alfredo Bernardo Poirier Schmitz}}
  % ORCID \texttt{0000-0003-2789-3630} \\
  % DNI \texttt{10803756}
\end{center}

\vspace*{0.5cm}

\begin{center}
  {
    \baselineskip=10pt
    \textbf{Lima, Per\'u} \\
    \textbf{Julio 2023}
  }
\end{center}
