\chapter*{Resumen}
Un tema central en la teor\'ia de n\'umeros es la distribuci\'on
de los n\'umeros primos sobre los enteros positivos.
En una direcci\'on,
de los trabajos de
Hadamard, de la Valle\'e Poussin and Newman,
nosotros sabemos que el PNT
(de su acrónimo en inglés \textit{Prime Number Theorem}, Teorema del N\'umero Primo)
es cierto por m\'etodos del an\'alisis complejo.
En otra direcci\'on,
Selberg, Breusch y Levinson probaron el PNT
v\'ia t\'ecnicas elementales, en el sentido de que solo usan an\'alisis real.
Hace menos de una d\'ecada,
Choudhary fortaleci\'o la prueba de Levinson.
Todas las pruebas elementales mencionadas
derivan el PNT usando la identidad de Selberg.

En esta tesis,
establecemos otra prueba para
la identidad de Selberg
m\'as simple que la de Choudhary en muchos aspectos
refinando los trabajos discutidos previamente.
Tambi\'en presentamos un algoritmo de tiempo lineal
para estimar una f\'ormula derivada de la identidad de Selberg.

\bigskip
\textbf{Palabras clave:}
Identidad de Selberg,
teor\'ia anal\'itica de n\'umeros,
teor\'ia algor\'itmica de n\'umeros.
