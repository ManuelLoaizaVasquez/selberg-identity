\chapter*{Resumen}
Un tema central en la Teor\'ia de N\'umeros es
la distribuci\'on de los n\'umeros primos sobre los enteros positivos.
En una direcci\'on, de los trabajos cl\'asicos de
Hadamard, Valle-Poussin y Newman,
nosotros sabemos que el TNP (Teorema del N\'umero Primo)
es cierto mediante el An\'alisis Complejo.
En otra direcci\'on,
Selberg, Erdős y Levinson
han probado el TNP
usando t\'ecnicas elementales en el sentido
de que solo usan An\'alisis Real.
Hace menos de una d\'ecada,
Choudhary ha probado elementalmente el TNP
fortaleciendo la prueba de Levinson.

En esta tesis,
establecemos nuevas pruebas para
la identidad de Selberg
y el TNP
refinando los trabajos mencionados
y desarrollamos algoritmos eficientes para estimar resultados num\'ericos.

\begin{itemize}
    \item
    \textbf{Nueva prueba elemental del Teorema del N\'umero Primo.}
    Siguiendo la estructura de las pruebas de Selberg y Choudhary,
    probamos el TNP sin utilizar los teoremas tauberianos de Shapiro.

    \item
    \textbf{Algoritmos de complejidad temporal polilogar\'itmica sublineal}
    Desarrollamos algoritmos de complejidad temporal
    \(\wt{O}(n^{3/4})\), \(\wt{O}(n^{2/3})\) y \(O(n)\)
    en el peor caso y los implementamos en C++
    para computar resultados intermedios y
    realizar un poco de experimentaci\'on num\'erica.
\end{itemize}

\bigskip
\textbf{Palabras clave:}
Teorema del N\'umero Primo,
Identidad de Selberg,
Teor\'ia de Cribas,
Teor\'ia Anal\'itica de N\'umeros,
Teor\'ia Algor\'itmica de N\'umeros.
